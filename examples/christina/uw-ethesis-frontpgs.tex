% T I T L E   P A G E
% -------------------
% This file goes along with the master LaTeX file uw-ethesis.tex
% Last updated May 27, 2009 by Stephen Carr, IST Client Services
% The title page is counted as page `i' but we need to suppress the
% page number.  We also don't want any headers or footers.
\pagestyle{empty}
\pagenumbering{roman}

\renewcommand{\contentsname}{Table of Contents}

% The contents of the title page are specified in the "titlepage"
% environment.
\begin{titlepage}
        \begin{center}
        \vspace*{1.0cm}

        \Huge
        {\bf Combinatorial and Probabilistic Approaches to Motif Recognition}

        \vspace*{1.0cm}

        \normalsize
        by \\

        \vspace*{1.0cm}

        \Large
        Christina Anne Boucher \\

        \vspace*{3.0cm}

        \normalsize
        A thesis \\
        presented to the University of Waterloo \\ 
        in fulfillment of the \\
        thesis requirement for the degree of \\
       	Doctor of Philosophy \\
        in \\
        Computer Science \\

        \vspace*{2.0cm}

        Waterloo, Ontario, Canada, 2010 \\

        \vspace*{1.0cm}

        \copyright\ Christina Anne Boucher 2010 \\
        \end{center}
\end{titlepage}

% The rest of the front pages should contain no headers and be numbered using Roman numerals starting with `ii'
\pagestyle{plain}
\setcounter{page}{2}

\cleardoublepage % Ends the current page and causes all figures and tables that have so far appeared in the input to be printed.
% In a two-sided printing style, it also makes the next page a right-hand (odd-numbered) page, producing a blank page if necessary.
 


% D E C L A R A T I O N   P A G E
% -------------------------------
  % The following is the sample Delaration Page as provided by the GSO
  % December 13th, 2006.  It is designed for an electronic thesis.
  \noindent
I hereby declare that I am the sole author of this thesis. This is a true copy of the thesis, including any required final revisions, as accepted by my examiners.

  \bigskip
  
  \noindent
I understand that my thesis may be made electronically available to the public.

\cleardoublepage
%\newpage

% A B S T R A C T
% ---------------

\begin{center}\textbf{Abstract}\end{center}

 
Short substrings of genomic data that are responsible for biological processes, such as gene expression, are referred to as motifs.  Motifs with the same function may not entirely match, due to mutation events at a few of the motif positions. Allowing for non-exact occurrences significantly complicates their discovery. Given a number of DNA strings, the motif recognition problem is the task of detecting motif instances in every given sequence without knowledge of the position of the instances or the pattern shared by these substrings.  

We describe a novel approach to motif recognition, and provide theoretical and experimental results that demonstrate its efficiency and accuracy. Our algorithm, MCL-WMR, builds an edge-weighted graph model of the given motif recognition problem and uses a graph clustering algorithm to quickly determine important subgraphs that need to be searched further for valid motifs. By considering a weighted graph model, we narrow the search dramatically to smaller problems that can be solved with significantly less computation. 

The {\sc Closest String} problem is a subproblem of motif recognition, and it is NP-hard.  We give a linear-time algorithm for a restricted version of the {\sc Closest String} problem, and an efficient polynomial-time heuristic that solves the general problem with high probability. We initiate the study of the smoothed complexity of the {\sc Closest String} problem, which in turn explains our empirical results that demonstrate the great capability of our probabilistic heuristic. Important to this analysis is the introduction of a perturbation model of the {\sc Closest String} instances within which we provide a probabilistic analysis of our algorithm.  The smoothed analysis suggests reasons why a well-known fixed parameter tractable algorithm solves {\sc Closest String} instances extremely efficiently in practice.  

Although the {\sc Closest String} model is robust to the oversampling of strings in the input, it is severely affected by the existence of outliers. We propose a refined model, the {\sc Closest String with Outliers} problem, to overcome this limitation.  A systematic parameterized complexity analysis accompanies the introduction of this problem, providing a surprising insight into the sensitivity of this problem to slightly different parameterizations. 

Through the application of probabilistic and combinatorial insights into the {\sc Closest String} problem, we develop sMCL-WMR, a program that is much faster than its predecessor MCL-WMR.  We apply and adapt sMCL-WMR and MCL-WMR to analyze the promoter regions of the canola seed-coat.  Our results identify important regions of the canola genome that are responsible for specific biological activities.  This knowledge may be used in the long-term aim of developing crop varieties with specific biological characteristics, such as being disease-resistant.

\cleardoublepage
%\newpage

    
     
% A C K N O W L E D G E M E N T S
% -------------------------------

\begin{center}\textbf{Dedication}\end{center}
This thesis would be incomplete without a mention of the support given by the members of the Waterloo Potters' Workshop.  Their encouragement and creativity has inspired me to share, explore and implement my ideas, whether it be on paper or with clay.   
\cleardoublepage

% A C K N O W L E D G E M E N T S
% -------------------------------

\begin{center}\textbf{Acknowledgements}\end{center}
This research project would not have been possible without the support of many people. I am extremely grateful to my supervisors, Ming Li and Prabhakar Ragde, who were abundantly helpful and offered invaluable assistance, support and guidance. Ming Li gave me with outstanding support for my research.  He provided me with great insight into my research topic, as well as fruitful collaborations; both helped broaden my view of the research field and led me to exciting research directions.  Prabhakar Ragde provided invaluable feedback that greatly helped improve my thesis.  His criticism taught me how to present my work in a clear, more concise manner. 

My deepest gratitude are also due to the members of the supervisory committee: Ming Li, Prabhakar Ragde, Anne Condon, Bin Ma, and Jonathan Buss.  Bin Ma's knowledge and insights in this research study were invaluable.  Discussions with him led to inspired exciting new research.   Anne Condon provided me with encouragement and support, and is an academic role-model for me.  Professors Charles Clarke, Naomi Nishimura, Nick Wormald, Tamer \"{O}zsu, and Yuying Li are also dearly mentioned. 

I am thankful to all my coauthors and collaborators.  I had the great opportunity to work with research members of Microsoft Research in Cambridge and the University of Manchester; notably, Christopher Bishop, John Winn, Markus Svens\'{e}n, Angela Smith, and Adnan Custovic.  I am grateful to my collaborators of the Alberta Research Council, Limin Wu and Saleh Shah, who provided a biological application of my algorithmic work.  I would like to acknowledge the administrative staff, especially Margaret Towell and Wendy Rush.

 I would also like to convey thanks to the Natural Sciences and Engineering Research Council (NSERC) of Canada, the Anita Borg Institute for Women and Technology, Google, and David R. Cheriton for providing the financial means and laboratory facilities.  This research was supported by the NSERC grants of Prabhakar Ragde, Dan Brown and Ming Li. 

I would like to thank all my friends for making my stay in Waterloo an enjoyable one.  I will fondly remember coffee breaks, pints of beer at Kick Offs and the Graduate House, GSA stuff, missing epic Halloween parties, knitting sessions, and the parties. Thanks to Kathleen Wilkie, Greg Zaverucha, Rob Warren, Cora Borradaile, Andrea Bunt, Steph Durocher, Jeff Dicker, Patrick Kling, Andrew Brown, Irene Pivotto, Peter Nelson, Robin Christian, Michael LaCroix, David Pritchard, Craig Sloss, Alex Hudek, Babak Alipanahi and Paul Church.  Montreal friends: Emily Austin, Christine Caruso and April Rose.  

I would like to thank my dear parents.  My father was the first to introduce me to University of Waterloo and the field of computer science. Lastly, I'd like to thank my best friend and husband, Jaime Ruiz, who has been a relentless source of support for me. He very generously took the time to listen to my presentations, proofread my papers and scholarship applications, and take care of household duties.

\cleardoublepage
%\newpage



% T A B L E   O F   C O N T E N T S
% ---------------------------------
\tableofcontents
\cleardoublepage
%\newpage

% L I S T   O F   T A B L E S
% ---------------------------
\listoftables
\addcontentsline{toc}{chapter}{List of Tables}
\cleardoublepage
%\newpage

% L I S T   O F   F I G U R E S
% -----------------------------
\listoffigures
\addcontentsline{toc}{chapter}{List of Figures}
\cleardoublepage
%\newpage

% L I S T   O F   S Y M B O L S
% -----------------------------
% \renewcommand{\nomname}{Nomenclature}
% \addcontentsline{toc}{chapter}{\textbf{Nomenclature}}
% \printglossary
%\cleardoublepage
% \newpage

% Change page numbering back to Arabic numerals
\pagenumbering{arabic}

