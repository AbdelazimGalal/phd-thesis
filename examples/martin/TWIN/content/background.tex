\section{Background}
\label{sec-background}

%\subsection{Definitions and notation}
\paragraph{Strings.}
Throughout we consider a string $\X = \X[1..n] = \X[1]\X[2]\ldots
\X[n]$ of $|\X| = n$ symbols drawn from the alphabet $[0..\sigma-1]$.
%We assume $\X[n]$ is a special ``end of string'' symbol, \$, smaller than
%all other symbols in the alphabet.
For $i=1,\ldots,n$ we
write $\X[i..n]$ to denote the {\em suffix} of $\X$ of length $n-i+1$,
that is $\X[i..n] = \X[i]\X[i+1]\ldots \X[n]$.  
%We will often refer to suffix $\X[i..n]$ simply as ``suffix $i$''. 
Similarly, we write
$\X[1..i]$ to denote the {\em prefix} of $\X$ of length $i$.
$\X[i..j]$ is the {\em substring} $\X[i]\X[i+1]\ldots \X[j]$ of $\X$
that starts at position $i$ and ends at $j$. 
%By $\X[i..j)$ we
%denote $\X[i..j-1]$.  If $j < i$ we
%define $\X[i..j]$ to be the empty string, also denoted by
%$\varepsilon$.

\paragraph{Optical Mapping.}
From a computational point of view, optical mapping is a process that takes two
strings: a genome $\A[1,n]$ and a restriction sequence $\B[1,b]$, and produces
an array (string) of integers $\M[1,m]$, such that $\M[i] = j$ if and only if 
$\A[j..j+b] = \B$ is the $i$th occurrence of $\B$ in $\A$.

For example, if we let $\B = \mbox{{\em act}}$ and 

\begin{center}
\begin{tabular}{p{0.4cm}*{22}{p{0.4cm}}}
& $\scriptstyle 1 $& $\scriptstyle 2 $& $\scriptstyle 3$& $\scriptstyle 4 $& $\scriptstyle 5 $& 
$\scriptstyle 6 $& $\scriptstyle 7 $& $\scriptstyle 8 $& $\scriptstyle 9 $& $\scriptstyle 10$&
$\scriptstyle 11 $& $\scriptstyle 12 $& $\scriptstyle 13$& $\scriptstyle 14 $& $\scriptstyle 15 $& 
$\scriptstyle 16 $& $\scriptstyle 17 $& $\scriptstyle 18 $& $\scriptstyle 19 $& $\scriptstyle 20$&
$\scriptstyle 21 $& $\scriptstyle 22 $\\
$\A $& $a$ & $t$ & $a$ & $c$ & $t$ & $t$ & $a$ & $c$ & $t$ & $g$ & $g$ 
&      $a$ & $c$ & $t$ & $a$ & $c$ & $t$ & $a$ & $a$ & $a$ & $c$ & $t$ \\
\end{tabular}
\end{center}

then we would have 
$$\M = 3,7,12,15,20.$$

It will also be convenient to view $\M$ slightly differently, as an array of fragment 
sizes, or distances between occurrences of $\B$ in $\A$ (equivalently differences
between adjacent values in $\M$). We denote this {\em fragment size domain} of $\M$, 
as the array $\F[1,m]$, defined such that $\F[i] = (\M[i]-\M[i-1])$, with $\F[1] = \M[1]-1$.  
Continuing with the example above, we have

$$\F = 2,4,5,3,5.$$

%TODO: 1) check the above makes sense 2) mention that in practice there are errors, 3) illustrate the alignment problem with an example.

\paragraph{Suffix Arrays.}
The suffix array~\cite{mm1993} $\SA_{\X}$ (we drop subscripts when
they are clear
from the context) of a string $\X$
is an array $\SA[1..n]$ which
contains a permutation of the integers $[1..n]$ such that $\X[\SA[1]..n]
< \X[\SA[2]..n] < \cdots < \X[\SA[n]..n]$.  In other words, $\SA[j] =
i$ iff $\X[i..n]$ is the $j^{\mbox{{\scriptsize th}}}$ suffix of $\X$
in lexicographical order.
%The inverse
%suffix array $\ISA$ is the inverse permutation of $\SA$, that is
%$\ISA[i] = j$ iff $\SA[j] = i$.
%Conceptually, $\ISA[i]$ tells us the position of suffix $i$ in $\SA$. 

\paragraph{SA Intervals.} 
For a string $\Y$, the $\Y$-interval in the suffix array $\SA_{\X}$ is
the interval $\SA[s..e]$ that contains all suffixes having $\Y$ as a
prefix. The $\Y$-interval is a representation of the occurrences of
$\Y$ in $\X$. For a character $c$ and a string $\Y$, the computation
of $c\Y$-interval from $\Y$-interval is called a \emph{left extension}.
%and the computation of $\Y$-interval from ${\Y}c$-interval is called a
%\emph{right contraction}. \emph{Left contraction} and \emph{right
%  extension} are defined symmetrically.

\paragraph{BWT and backward search.}
The Burrows-Wheeler Transform~\cite{bw1994} $\BWT[1..n]$ is a
permutation of $\X$ such that $\BWT[i] = \X[\SA[i]-1]$ if $\SA[i]>1$
and $\$$ otherwise. We also define $\LF[i] = j$ iff $\SA[j] =
\SA[i]-1$, except when $\SA[i] = 1$, in which case $\LF[i] = I$,
where $\SA[I] = n$.

Ferragina and Manzini~\cite{fm2005} linked $\BWT$ and $\SA$ in the
following way.  
Let $\C[c]$, for symbol $c$, be the number of symbols
in $\X$ lexicographically smaller than $c$.  The function
$\rank(\X,c,i)$, for string $\X$, symbol $c$, and integer $i$, returns
the number of occurrences of $c$ in $\X[1..i]$.  It is well known that
$\LF[i] = \C[\BWT[i]] + \rank(\BWT,\BWT[i],i)$.  Furthermore, we can
compute the left extension using $\C$ and $\rank$.  If $\SA[s..e]$ is
the $\Y$-interval,
%containing all the suffixes prefixed with string $\Y$, 
then
$\SA[\C[c]+\rank(\BWT,c,s),\C[c]+\rank(\BWT,c,e)]$ is
the $c\Y$-interval.
This is called \emph{backward search}~\cite{fm2005}, and a data
structure supporting it is called an {\em FM-index}.

%\paragraph{Wavelet Trees and approximate pattern matching.}
