\section{The Pairwise Rmap Alignment Problem}

%\subsection{Rmap Data Description}

%estriction mapping (by optical or other means) can be seen as a process that takes in twosequences: a genome $\A[1,n]$ and a restriction enzyme's recognition sequence $\B[1,b]$, and produces an array (sequence) of integers  $\C$, the {\em genome restriction map}, which we define as follows. First define the array of integers $\C[1,m]$ where $\C[i] = j$ if and only if  $\A[j..j+b] = \B$ is the $i$th occurrence of $\B$ in $\A$.Then $\R[i] = (\C[i]-\C[i-1])$, with $\R[1] = \C[1]-1$.
%In words, $\R$ contains the distance between occurrences of $\B$ in $\A$.

Given a genome $\A[1,n]$ and a restriction enzyme's site $\B[1,b]$, the optical mapping produces Rmaps, which are arrays of lengths---or fragment sizes---between occurrences of $\B$ in $\A$. The background section provides details on the optical mapping process.  Producing Rmap data is an error prone process. Thus, there are three types of errors can occur: (1) missing and false cuts that delimit fragments; (2) missing fragments; and (3) inaccuracy in the fragment sizes.  For example, let $\R = 2,4,5,3,5$ be an error-free Rmap, then an example of an Rmap with the first type of error could be $\R' = 6,5,3,5$ (the first cut site is missing so the fragment sizes 2, and 4 are summed to become 6 in $\R'$); an example of a Rmap with the second type of error would be $\R'' = 2,4,3,5$ (the third fragment is missing); and lastly, the third type of error could be illustrated by $\R''' = 2,4,7,3,5$ (the size of the third fragment is inaccurately given)\footnote{See appendix for a description of the frequency to which these errors occur.}.
%Hence, a more accurate Rmap for this small example would be something like $\R = 7,6,3,4.$  

%\subsection{Problem Definition}

The pairwise Rmap alignment problem aims to align one Rmap (the \emph{query}) $\R_q$ against the set of all other Rmaps in the dataset (the \emph{target}). We denote the target database as $\R_1 \ldots \R_n$, where each $\R_i$  is a sequence of fragment sizes, i.e, $\R_i = [f_{i1}, .., f_{im_i}]$.  An alignment between two Rmaps is a relation between them comprising groups of zero or more consecutive fragment sizes in one Rmap to groups of zero or more consecutive fragments in the other.  For example, given $\R_i =  [4, 5, 10, 9, 3]$ and $\R_j = [10, 9, 11]$ one possible alignment is $\{[4,5], [10]\}, \{ [10], [9]\}, \{[9], [11]\}, \{[3], []\}$.  A group may contain more than one fragment (e.g. $[4,5]$) when the restriction site delimiting the fragments is absent in the corresponding group of the other Rmap (e.g $[10]$). This can occur if there is a false restriction site in one Rmap, or there is a missing restriction site in the other.  Since we cannot tell from only two Rmaps which of these scenarios occurred, for the purpose of our remaining discussion it will be sufficient to consider only the scenario of missed (undigested) restriction sites. %, given that they arrise in both target and query Rmaps.

%During the alignment search process, we try various assignments of fragments to groups. Since groups need to agree well in size, it is useful to have a concept that ignores the number of constituent fragments.  Thus, we introduce the term {\em compound fragment} to refer to a hypothetical fragment of the total length of a putative group. % We refer to a \emph{proper} compound fragment if it is specifically composed of multiple fragments.  In our previous example, the group $[4,5]$ whose sum is 9 is a proper compound fragment and would pair with the compound fragment 10 in $\R_j$. 

