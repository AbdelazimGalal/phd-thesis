\chapter{Conclusion}
\label{sec:conclusion}

In this thesis, we proposed the first Burrows--Wheeler based representation of de Bruijn graphs, a linchpin in bioinformatic pipelines. This reduced the memory requirements by an order of magnitude, enabling basic DNA assembly to be performed on a laptop.

While there are competing representations with similar memory benefits, the ordered nature of the Burrows--Wheeler based representation enabled augmentation using additional data structures to support more powerful operations.

In the case of the variable order de Bruijn graph, by including the longest common suffix array, stored using a wavelet tree, we can quickly change the value of $k$ on the fly. This is the first data structure to support this, and can be used to improve genome assembly quality, while taking only 3.5 times the space and 30\% longer to construct than a single de Bruijn graph, and avoiding building multiple de Bruijn graphs like previous approaches would take.

In the case of the Colored de Bruijn graph, we utilised a compressed bit matrix to store color information in the same order as the edges. This allowed us to implement the algorithms described in~\cite{ICTFM12} in almost two orders of magnitude less space, ultimately enabling this analysis for datasets that were previously unsupported. Finally, we demonstrated its practical use by applying it to locating bacterial outbreaks in food supply chains.

After publishing these papers, some open questions remained:

\begin{enumerate}
\item The Variable-Order de Bruijn Graph described how to change the value of $k$ on the fly. This could in theory support any procedure utilising multiple $k$ values during traversal, but the exact methodology was left unanswered. Using the Variable-Order de Bruijn graph, how should order change operations be applied to improve the quality of the assembly? How much could the quality be improved?
  
\item A method using Range Minimum Query data structures to support varying $k$ was also suggested, which would take a performance hit, yet reduce the size. However, experimental results were only given for the implementation using a Wavelet Tree. It would be interesting to see how they compare in practice.
  
\item The Colored de Bruijn Graph becomes more powerful when built with data from larger populations. This is good news, as sequence data is being produced at an increasing rate. However, as the graph is a static data structure, it must be rebuilt using the original $k$-mers, and new $k$-mers, whenever new sequence data is added. There is ongoing research into merging Burrows--Wheeler transforms -- could this be utilised to allow updating a Colored de Bruijn Graph?
  
\item Our color matrix implementation was rudimentary, using only a general purpose succinct bit vector data structure. It should be possible to improve the compression by using an approach specifically suited to the task.
\end{enumerate}

Since the papers in this thesis were published, there has been a number of papers published which further explored the power of the Burrows--Wheeler based representation, including addressing the above questions.

Belazzougui et al. were able to improve the complexity of variable-order backward traversal, which was the slowest of our proposed operations~\cite{Bel18}. This would allow it to be practical to store only one of the reverse complements of the DNA, in order to reduce the memory it takes, while taking a smaller speed penalty.

Díaz-Domínguez et al. describe a variant of the variable-order graph, called the hidden-order de Bruijn graph, replacing the LCP array with a succinct cartesian tree, which reduced the memory requirements, and allowed them to develop a scheme to extract remarkably long contigs, which are more informative than what would be available in a simple fixed-$k$ de Bruijn graph~\cite{Dia18}. Later, Díaz-Domínguez et al. further augmented this to simulate a complete DNA string graph (briefly described in the introduction of this thesis), unlocking substantially longer contigs, still~\cite{Dia19}.

Muggli et al. presented a merging algorithm, enabling two Colored de Bruijn graphs to be combined into one -- we could now re-use existing graphs while gathering new sequence data~\cite{Mug19}. This was shortly after improved on by Egidi et al., who reduced the working space by half~\cite{Egi19}. There has also been research into improving the compression scheme for the color matrix~\cite{Ali18,Mus18}.

There have also been a number of papers which augment the succinct de Bruijn graph with other additional information, such as long range distance information~\cite{Tur18} and optical maps~\cite{Muk18}, providing more positional information to better disambiguate contigs.

Other research has included making the de Bruijn graph fully dynamic~\cite{Cra18,Bel16}, and applying colored de Bruijn graphs to real-time search of increasingly large bacterial and viral genomic databases~\cite{Bra17}. A review is also available in~\cite{Mar19}.

By introducing a simple, customizable, succinct representation of the all-important de Bruijn graph, we are now able to better analyze increasingly large data, unearthing deeper understandings of DNA, and how to manipulate it. With the advent of CRISPR -- the high precision DNA editing tool -- such knowledge can now be applied to program organisms as we see fit. We can improve ourselves how we see fit, create new life forms, and build defenses against illnesses. Evolutionary selection, once natural, will become our prerogative.

%\section{Future Work}
%\label{sec:future-work}
