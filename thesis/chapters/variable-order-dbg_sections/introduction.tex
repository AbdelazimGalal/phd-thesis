\section{Introduction}\label{sec:introduction}
% Computer Society journal (but not conference!) papers do something unusual
% with the very first section heading (almost always called "Introduction").
% They place it ABOVE the main text! IEEEtran.cls does not automatically do
% this for you, but you can achieve this effect with the provided
% \IEEEraisesectionheading{} command. Note the need to keep any \label that
% is to refer to the section immediately after \section in the above as
% \IEEEraisesectionheading puts \section within a raised box.


% The very first letter is a 2 line initial drop letter followed
% by the rest of the first word in caps (small caps for compsoc).
% 
% form to use if the first word consists of a single letter:
% \IEEEPARstart{A}{demo} file is ....
% 
% form to use if you need the single drop letter followed by
% normal text (unknown if ever used by the IEEE):
% \IEEEPARstart{A}{}demo file is ....
% 
% Some journals put the first two words in caps:
% \IEEEPARstart{T}{his demo} file is ....
% 

Accurate assembly of genomes is a fundamental problem in
bioinformatics and is vital to several ambitious scientific projects, including
the 10,000 vertebrate genomes (Genome 10K)~\cite{Haussler:2009}, \emph{Arabidopsis}
variations (1001 genomes)~\cite{Ossowski08}, human variations (1000
genomes)~\cite{Abecasis12}, and the Human Microbiome~\cite{hmp} projects. The
genome assembly process builds long contiguous DNA sequences, called {\em
contigs}, from shorter DNA fragments, called {\em reads}, typically 100-150
(DNA) symbols in length.


In Eulerian sequence assembly \cite{IW95,PTW}, a {\em de Bruijn graph} is
constructed with a vertex $v$ for every $K$-mer (substring of length $K$)
present in an input set of reads, and an edge $(v, v')$ for every observed $(K +
1)$-mer in the reads with $K$-mer prefix $v$ and $K$-mer suffix $v'$.  Contigs
are then extracted from this graph.  Most state-of-the-art assemblers use this
paradigm~\cite{bankevich2012spades,peng2010idba,Li:2010,Simpson:2009,Butler:2008}, 
%including SPAdes \cite{bankevich2012spades}, IDBA \cite{peng2010idba}, Euler-SR
%\cite{Chaisson:2008}, Velvet \cite{Zerbino:2008}, SOAPdenovo \cite{Li:2010},
%ABySS \cite{Simpson:2009} and ALLPATHS \cite{Butler:2008}.  De Bruijn graph
%based assemblers 
and follow the same general outline: extract $(K + 1)$-mers from the reads;
construct the de Bruijn graph on the set of $(K +1)$-mers; simplify the graph; 
%resolve the repeated regions by using mate-pair information, 
and construct the contigs (simple paths in the graph). The value of $K$ can be,
and is often required to be, specified by the user.

Determining an appropriate value of $K$ is important and has a direct impact on
assembly quality. Stated very briefly, when $K$ is too small the resulting graph
is complicated by spurious edges and nodes, and when $K$ is too large the graph
becomes too sparse and possibly disconnected.
%; see Figure~\ref{}.  Repetitive regions in the underlying genome are
%especially problematic since they give rise to spurious edges and nodes in the
%de Bruijn graph \cite{alkan2011limitations} and are very sensitive to the
%choice of $K$.

%A perhaps obvious combinatorial approach for building a simplified de Bruijn
%graph would be to allow the value of $K$ to vary; having a larger value of $K$
%where the graph is more complex and a smaller value of $K$ where the graph
%would likely contain fewer spurious edges and f.  
%%A major difficulty in implementing this approach is determining a practical method that makes this idea feasible when assembling large genomes.  
In an attempt to circumvent the need to choose a single, ideal value of $K$,
SPAdes~\cite{bankevich2012spades} and IDBA~\cite{peng2010idba} use a number of
different $K$ values.
%Peng et al.~\cite{peng2010idba} and Bankevich et al.~\cite{bankevich2012spades}
%both introduced assemblers that use a number of different $k$ values. 
IDBA~\cite{peng2010idba} builds a number of de Bruijn graphs for each a fixed
set of $K$ values.  At a given iteration of the algorithm, the de Bruin graph
for the current value of $K$ is built from the reads and the contigs for that
  graph are constructed, then all the reads that align to at least one of those
  contigs are removed from the current set of reads. In the next iteration the
  graph is built by converting every edge from the previous graph to a vertex
  while treating contigs as edges. SPAdes \cite{bankevich2012spades} uses a
    similar approach but uses all the reads at each iteration.  

\subsection{Our Contribution} SPAdes~\cite{bankevich2012spades} and
IDBA~\cite{peng2010idba} represent the state-of-the-art for genome assemblers,
producing assemblies of greatly improved quality compared to previous
approaches. However, their need to construct several de Bruijn graphs of
different orders over the assembly process makes them extremely slow on large
genomes.

In this paper we address this problem by describing a succinct data
structure that, for a given $K$, efficiently represents {\em all} the de Bruijn
graphs for $k \le K$ and allows navigation within and between each graph. In addition
also describe an alternative representation which is smaller but slower.

We have implemented the faster version of our data structure and shown that in practice
it requires around $3.5$ times the space of a graph for a single $K$, and incurs a
modest slow down in construction time and on navigation operations. Compared
with the conference version of this paper~\cite{varorder-dcc}, we have implemented an
external memory construction algorithm, and demonstrated the scalability of our
structure on much larger data sets.

%While this method has been shown to greatly improve assembly quality
%\cite{bankevich2012spades,peng2010idba}, there currently are not any space
%efficient data structures for multiple de Bruijn graphs of different orders
%(i.e. different values of $K$).  This is the problem we address in this paper.  

\subsection{Related Work} There are several succinct data structures for the de
Bruijn graph of a single order (i.e.~value of $K$).  One of the first approaches
was introduced by Simpson et al.~\cite{Simpson:2009} as part of the development
of the ABySS assembler.  Their method stores the graph as a distributed hash
table and thus requires 336 GB to store the graph corresponding to a set of
reads from a human genome (HapMap: NA18507).  In 2011, Conway and
Bromage~\cite{conway} reduced space requirements by using a sparse bitvector (by
Okanohara and Sadakane~\cite{bitvector}) to represent the $(K + 1)$-mers (the
edges), and used rank and select operations (to be described shortly) to
traverse it. As a result, their representation took 32 GB for the same data set.
Minia, by Chikhi and Rizk~\cite{wabi}, uses a Bloom filter to store edges.
%(with additional structures to avoid false positive edges that would affect the
%assembly). 
They traverse the graph by generating all possible outgoing edges at each node
and testing their membership in the Bloom filter. Using this approach, the graph
was reduced to 5.7 GB on the same dataset.  Contemporaneously, Bowe, Onodera,
Sadakane and Shibuya~\cite{BOSS12} developed a different succinct data structure
based on the Burrows-Wheeler transform~\cite{BW94} that requires 2.5 GB.
Their representation, which henceforth we refer to as BOSS from the authors'
initials, is a starting point for our methods and we will discuss it in detail
below.  Very recently Chikhi {et al.}~\cite{paul} implemented the de Bruijn
graph using an FM-index and {\em minimizers}.  Their method uses 1.5 GB on the
same NA18507 data.  

The data structure of Bowe et al.~\cite{BOSS12}
%and BWT construction of short reads of Liu et al.~\cite{liu_et_al} --- are 
is combined with ideas from IDBA-UD~\cite{idbaud} in a metagenomics assembler
called MEGAHIT~\cite{megahit}.  In practice MEGAHIT requires more memory than
competing methods 
%(345G in comparison to 287G and 29G on one dataset) 
but produces significantly better assemblies.
%on metagenomics data.

Lastly, Lin and Pevzner~\cite{mdbg} recently introduced the {\em manifold de
Bruijn graph}, which 
%is a generalization of the traditional de Bruijn graph that effectively allows
%$K$ to vary by 
associates arbitrary substrings with nodes (the substrings are fixed during
preprocessing), rather than $K$-mers. Lin and Pevzner's structure is mainly of
theoretical interest since it has not yet been implemented.
%; there's no implementation of the manifold de Bruijn graph and cannot handle
%errors within the data.  The succinct data structure we present may lead to
%insights as to how to construct the manifold de Bruijn graph in a space and
%time efficient manner.

\subsection{Roadmap} Section \ref{sec:preliminaries} sets notation, and formally
lays down the problem and auxiliary data structures we use.
Section~\ref{sec:BOSS} gives details of the BOSS representation~\cite{BOSS12}.
Sections~\ref{sec:changing} and~\ref{sec:implementing} then describes our
variable-order de Bruijn graph structure. In Section~\ref{sec:experiments} we
report on experiments comparing the practical performance of our data structure
to that of a single-order de Bruijn graph. Section~\ref{sec:conclusion} offers
directions for future work.
%areas of research that warrant investigation.

