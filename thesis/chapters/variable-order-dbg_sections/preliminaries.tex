\section{Preliminaries} \label{sec:preliminaries}

%DO WE WANT A FORMAL DEFINITION OF THE DEBRUIJN GRAPH IN HERE?
\subsection{De Bruijn Graphs} \label{sec:dbg}

Given an alphabet $\Sigma$ of $\sigma$ symbols and a set of strings $\lbrace
S_1, S_2, \ldots, S_t \rbrace$, $S_i \in \Sigma^{+}$, the {\em de Bruijn graph}
of order $K$, denoted $G^S_K$, or just $G_K$, when the context is clear, is a
directed, labelled graph defined as follows.

Let $M_{K}$ be the set of distinct $K$-mers (strings of length $K$) that occur
as substrings of some $S_i$. $M_{K+1}$ is defined similarly.  $G_K$ has exactly
$|M_{K}|$ nodes and with each node $u$ we associate a distinct $K$-mer from
$M_{K}$, denoted $\nodelabel(u)$. Edges are defined by $M_{K+1}$: for each
string $T \in M_{K+1}$ there is a directed edge, labelled with symbol $T[K+1]$,
from node $u$ to node $v$, where $\nodelabel(u) = T[1,K]$ and $\nodelabel(v) =
T[2,K+1]$. 

\subsection{Rank and Select} \label{sec:rank} Two basic operations used in almost
every succinct and compressed data structure are {\em rank} and {\em select}.
Given a sequence (string) $S[1,n]$ over an alphabet $\Sigma =
\{1,\ldots,\sigma\}$, a character $c \in \Sigma $, and integers $i$,$j$,
$\rank_c(S,i)$ is the number of times that $c$ appears in $S[1,i]$, and
$\select_c(S,j)$ is the position of the $j$-th occurrence of $c$ in $S$.
%There is a great variety of techniques to answer these queries, with
%suitability depending on the nature of the sequence, for example, on whether or
%not it will be compressed and on the size of the alphabet.
For a binary string $B[1,n]$, the classic solution for rank and
select~\cite{Mun96} is built upon the input sequence, requiring $o(n)$
additional bits.  Generally, $\rank_1$ and $\select_1$ are considered the
default rank and select queries.  More advanced solutions
(e.g.~\cite{bitvector}) achieve zero-order compression of $B$,
%For example, the several structures (e.g.~\cite{bitvector}), (see
%also~\cite{kkp2014}), 
representing it in just $nH_0(B) + o(n)$ bits of space, and supporting $\rank$
and $\select$ operations in constant time. 
%Several practical implementations and improvements of RRR exists (see,
%e.g.,~\cite{kkp2014}).

\subsection{Wavelet Trees} \label{sec:WVT} To support rank and select on larger
alphabet strings, the wavelet tree~\cite{ggv2003,n2013} is a commonly used data
structure that occupies $n\log\sigma + o(n\log\sigma)$ bits of space and
supports $\rank$ and $\select$ queries in $\Oh{\log\sigma}$ time.  Wavelet trees
also support a variety of more complex queries on the underlying string (see,
e.g.~\cite{gnp2012}), in $\Oh{\log\sigma}$ time, and we will make use of some of
this functionality in Section~\ref{sec:implementing}.
%One we will make use of in this paper is the {\em range successor query},
%$\rsucc_c(i,j)$, which returns the smallest character $d > c$ in $S[i,j]$, or
%$\inf$ if no such character exists.  Wavelet trees support \rsucc queries in
%$\Oh{\log\sigma}$ time.


%We will also make use of the Wavelet Tree~\cite{GGV03} data structure on a
%string $T=a_{1}a_{2} \ldots a_{n}$ over an alphabet $\Sigma$.
%
%The Wavelet Tree of $T$ is a binary balanced tree, where each leaf represents a
%symbol of $\Sigma$. The root is associated with the complete sequence $T$. Its
%left child is associated with a subsequence obtained by concatenating the
%symbols $a_i$ of $T$ satisfying $a_i < \sigma /2$. The right child corresponds
%to the concatenation of every symbol $a_i$ satisfying $a_i \geq \sigma$.  This
%relation is maintained recursively up to the leaves, which will be associated
%with the repetitions of a unique symbol.  At each node we store only a binary
%sequence of the same length of the corresponding sequence, using at each
%position a $0$ to indicate that the corresponding symbol is mapped to the left
%child, and a $1$ to indicate the symbol is mapped to the right child.
%
%If the bitmaps of the nodes support constant-time {\em rank} and {\em select}
%queries, then the Wavelet Tree support fast $access$, {\em rank} and {\em
%select} on $T$.
%
%\emph{Access:} In order to obtain the value of $a_{i}$ the algorithm begins at
%the root, and depending on the value of the root bitmap $B$ at position $i$, it
%moves down to the left or to the right child. If the bitmap value is $0$ it
%goes to the left, and replaces $i \leftarrow \rank_{0}(B,i)$. If the bitmap
%value is $1$ it goes to the right child and replaces $i \leftarrow
%\rank_{1}(B,i)$.  When a leaf is reached, the symbol associated with that leaf
%is the value of $a_i$.
%
%\emph{Rank:} To obtain the value of $\rank_c(S,i)$ the algorithm is similar: it
%begins at the root, and goes down updating $i$ as in the previous query, but
%the path is chosen according to the bits of $c$ instead of looking at $B[i]$.
%When a leaf is reached, the $i$ value is the answer.
%
%\emph{Select:} The value of $\select_c(S,j)$ is computed as follows: The
%algorithm begins in the leaf corresponding to the character $c$, and then moves
%upwards until reaching the root.  When it moves from a node to its parent, $j$
%is updated as $j \leftarrow \select_{0}(B,j)$ if the node is a left child, and
%$j \leftarrow \select_{1}(B,j)$ otherwise. When the root is reached, the final
%$j$ value is the answer.
%
%{\em Range Successor Queries}.


