\section{Experiments}
\label{sec:experiments}

\begin{table*}[!t]
%\caption{Construction Time and Memory Usage (top), and Mean Time per Navigation Operation (lower).}
\caption{Input size (top), construction time, memory use, and structure size (middle), and mean time taken for each 
navigation operation (lower), for all data sets and both structures. For variable-order, the multipliers in
parenthesis are the increase over the fixed-order results. Cells marked ``N/A'' for fixed-order indicate operations not
possible with that structure.
%The times in parentheses for $\longer$ are the mean times per node in the resulting set.
}
\begin{tabularx}{\linewidth}{rd{3.2}d{4.9}d{3.2}d{4.9}d{3.2}d{5.9}d{3.2}d{5.9}}
% manual : http://ftp.jaist.ac.jp/pub/CTAN/macros/latex/required/tools/tabularx.pdf
						%\cline{2-9}
\toprule
Dataset     & \multicolumn{2}{c}{{\em E.~coli}} & \multicolumn{2}{c}{Human chromosome 14} & \multicolumn{2}{c}{Human} 		& \multicolumn{2}{c}{Parrot} \\
% I think DSK size + number of K-mers is enough to demonstrate the increasing data set size
% I don't have times for DSK, so I'd have to run those again if needed
%Genome Size (bp) & \multicolumn{2}{c}{{4,639,221}} & \multicolumn{2}{c}{88,289,540} & \multicolumn{2}{c}{} 		& \multicolumn{2}{c}{} \\
%Number of Reads & \multicolumn{2}{c}{{}} & \multicolumn{2}{c}{} & \multicolumn{2}{c}{} 		& \multicolumn{2}{c}{} \\
%\midrule
%K & \multicolumn{2}{c}{{27}} & \multicolumn{2}{c}{55} & \multicolumn{2}{c}{55} 		& \multicolumn{2}{c}{55} \\
%Frequency Threshold & \multicolumn{2}{c}{{1}} & \multicolumn{2}{c}{1} & \multicolumn{2}{c}{2} 		& \multicolumn{2}{c}{1} \\
%DSK Time (mins) & \multicolumn{2}{c}{{}} & \multicolumn{2}{c}{} & \multicolumn{2}{c}{} 		& \multicolumn{2}{c}{} \\
DSK Size (GB) & \multicolumn{2}{c}{{1.52}} & \multicolumn{2}{c}{6.88} & \multicolumn{2}{c}{26.74} 		& \multicolumn{2}{c}{70.28} \\
Number of $K$-mers & \multicolumn{2}{c}{{204,098,902}} & \multicolumn{2}{c}{461,445,333} & \multicolumn{2}{c}{1,794,522,954} 		& \multicolumn{2}{c}{4,716,731,435} \\
BOSS Order & \multicolumn{1}{c}{fixed} 	& \multicolumn{1}{c}{variable} & \multicolumn{1}{c}{fixed} 	& \multicolumn{1}{c}{variable} & \multicolumn{1}{c}{fixed} 	& \multicolumn{1}{c}{variable} & \multicolumn{1}{c}{fixed} 	& \multicolumn{1}{c}{variable} \\
\midrule
%\cline{1-9}
Construction (mins) & 3.93 & 5.09 \enspace (1.30{\sf x}) & 14.37 & 18.72 \enspace (1.30{\sf x}) & 64.45 & 83.85 \enspace (1.30{\sf x})& 162.58 & 225.73 \enspace (1.39{\sf x})\\
Graph Size (GB)  			   & 0.16  & 0.41 \enspace (2.56{\sf x})  & 0.40   & 1.38 \enspace (3.45{\sf x}) & 1.67 & 5.42 \enspace (3.25{\sf x}) & 4.20 & 13.60 \enspace (3.24{\sf x}) \\
Peak RAM (GB)  		 & 3.16 & 3.16 \enspace (1.00{\sf x}) & 3.22 & 3.22 \enspace (1.00{\sf x})& 7.65 & 9.31 \enspace (1.22{\sf x}) & 15.30 & 15.29 \enspace (1.00{\sf x}) \\
Peak Disk (GB)  	 & 12.17 & 12.17 \enspace (1.00{\sf x}) & 56.68 & 56.68 \enspace (1.00{\sf x}) & 248.37 & 248.37 \enspace (1.00{\sf x}) & 562.28 & 562.28 \enspace (1.00{\sf x})\\
%\cline{1-9}
\midrule
$\forward$ ($\mu$s)   & 6.00 & 17.03 \enspace (2.84{\sf x}) &6.24	&16.17 \enspace (2.59{\sf x}) &7.07	&18.31 \enspace (2.59{\sf x})&7.77	 &19.39 \enspace (2.50{\sf x})\\
$\backward$ ($\mu$s)  & 8.23 & 59.77 \enspace (7.26{\sf x}) &8.47	&55.63 \enspace (6.57{\sf x}) &9.27	&62.85 \enspace (6.78{\sf x})&10.46 &63.87 \enspace (6.11{\sf x})\\
$\lastchar$ ($\mu$s)  & 0.01 &  0.01 \enspace (1.00{\sf x}) &0.01	& 0.01 \enspace (1.00{\sf x}) &0.01	& 0.01 \enspace (1.00{\sf x})&0.01	 &0.01 \enspace (1.00{\sf x})\\

$\maxlen$ ($\mu$s)    &\multicolumn{1}{r}{N/A}  & 1.43   &\multicolumn{1}{r}{N/A} &1.56	   &\multicolumn{1}{r}{N/A}  	&2.02	    &\multicolumn{1}{r}{N/A}  &2.46 \\ 
$\maxlen_c$ ($\mu$s)  &\multicolumn{1}{r}{N/A}  &	5.41  &\multicolumn{1}{r}{N/A} &5.98	   &\multicolumn{1}{r}{N/A}  &6.71	    &\multicolumn{1}{r}{N/A}  &7.49 \\
$\shorter_1$ ($\mu$s) &\multicolumn{1}{r}{N/A}   &	14.65 &\multicolumn{1}{r}{N/A} &17.72	 &\multicolumn{1}{r}{N/A} 	  &19.54	  &\multicolumn{1}{r}{N/A}  &19.84 \\
$\shorter_2$ ($\mu$s) &\multicolumn{1}{r}{N/A}  &	14.83 &\multicolumn{1}{r}{N/A} &17.79	 &\multicolumn{1}{r}{N/A}  	&19.68	  &\multicolumn{1}{r}{N/A}  &19.98 \\
$\shorter_4$ ($\mu$s) &\multicolumn{1}{r}{N/A}   &	15.11 &\multicolumn{1}{r}{N/A} &18.02	 &\multicolumn{1}{r}{N/A}   &19.90	  &\multicolumn{1}{r}{N/A}  &20.20 \\
$\shorter_8$ ($\mu$s) &\multicolumn{1}{r}{N/A}   &	15.73 &\multicolumn{1}{r}{N/A} &18.39	 &\multicolumn{1}{r}{N/A}   &20.29	  &\multicolumn{1}{r}{N/A}  &20.64 \\
%$\shorter$ ($\mu$s)   &\multicolumn{1}{r}{N/A}   &	15.08 &\multicolumn{1}{r}{N/A} &17.98	 &\multicolumn{1}{r}{N/A}   &19.85	  &\multicolumn{1}{r}{N/A}  &20.17 \\ % mean avg of above 4
$\longer_1$ ($\mu$s)  &\multicolumn{1}{r}{N/A}   &21.53   &\multicolumn{1}{r}{N/A} &18.61	 &\multicolumn{1}{r}{N/A}   &21.06	  &\multicolumn{1}{r}{N/A}  &20.57 \\ 
$\longer_2$ ($\mu$s)  &\multicolumn{1}{r}{N/A}  &56.96   &\multicolumn{1}{r}{N/A} &41.08	 &\multicolumn{1}{r}{N/A}   &49.01	  &\multicolumn{1}{r}{N/A}  &47.07\\
$\longer_4$ ($\mu$s)  &\multicolumn{1}{r}{N/A}   &503.60  &\multicolumn{1}{r}{N/A} &323.50	 &\multicolumn{1}{r}{N/A}   &446.51	  &\multicolumn{1}{r}{N/A}  &428.97 \\
$\longer_8$ ($\mu$s)  &\multicolumn{1}{r}{N/A}  &6441.33 &\multicolumn{1}{r}{N/A}  &5338.38 &\multicolumn{1}{r}{N/A}   &18349.80	&\multicolumn{1}{r}{N/A}  &24844.80 \\

%$\longer_1$ ($\mu$s)  &\multicolumn{1}{r}{N/A} &21.53 \enspace (10.82)  &\multicolumn{1}{r}{N/A} &18.61 \enspace (12.38)  &\multicolumn{1}{r}{N/A} &21.06 \enspace (13.25)    &\multicolumn{1}{r}{N/A} &20.57 \enspace (12.99)\\ 
%$\longer_2$ ($\mu$s)  &\multicolumn{1}{r}{N/A} &56.96 \enspace (8.74)   &\multicolumn{1}{r}{N/A} &41.08 \enspace (11.24)  &\multicolumn{1}{r}{N/A} &49.01 \enspace (11.74)    &\multicolumn{1}{r}{N/A} &47.07 \enspace (11.02)\\
%$\longer_4$ ($\mu$s)  &\multicolumn{1}{r}{N/A} &503.60 \enspace (7.98)  &\multicolumn{1}{r}{N/A} &323.50 \enspace (10.59) &\multicolumn{1}{r}{N/A} &446.51 \enspace (11.04)   &\multicolumn{1}{r}{N/A} &428.97 \enspace (9.95)\\
%$\longer_8$ ($\mu$s)  &\multicolumn{1}{r}{N/A} &6441.33 \enspace (6.44) &\multicolumn{1}{r}{N/A} &5338.38 \enspace (9.03) &\multicolumn{1}{r}{N/A} &18349.80 \enspace (10.09) &\multicolumn{1}{r}{N/A} &24844.80 \enspace (9.48)\\

%$\longer_1$ nodes&	N/A	39.80	N/A	30.06	N/A	31.78	N/A	31.66
%$\longer_2$ nodes&	N/A	130.32	N/A	73.11	N/A	83.48	N/A	85.42
%$\longer_4$ nodes&	N/A	1262.84	N/A	610.77	N/A	809.06	N/A	861.91
%$\longer_8$ nodes&	N/A	20010.15	N/A	11824.04	N/A	36370.68	N/A	52388.38
\bottomrule
%\cline{2-9}
\end{tabularx}
% $\dagger$ $\lastchar$ is a reverse lookup in a very small array, so the speed is in fractions of a nanosecond.
% $\ddagger$ $\longer$ is reported as the average time  for the inputs $1$,$2$,$4$,$8$, due to much longer calculation time. }
\label{tab:nav-time}
\end{table*}

%We have implemented the faster version of our data structure on top of an efficient implementation of the BOSS single-$K$ data structure\footnote{The
We have implemented the wavelet tree based data structure on top of an efficient implementation of the BOSS single-$K$ data structure\footnote{The
implementation is released under GPLv3 license at \url{http://github.com/cosmo-team/cosmo}. As Cosmo is under continuous development,
a static snapshot of the code used in this paper is available at \url{https://github.com/cosmo-team/cosmo/tree/varord-paper}.}.
Both structures make use of the SDSL-lite software library\footnote{\url{https://github.com/simongog/sdsl-lite}} for succinct data structures, and the
the construction code makes use of the STXXL software library\footnote{\url{https://github.com/stxxl/stxxl}} for external memory data structures and sorting.
The construction code is also concurrent in many places.%, making use of C++11 threads, and OpenMP\footnote{\url{http://openmp.org/}} for parallel internal memory sorting.
The smaller but slower version was not implemented.
%Further implementation details are described in \ref{sec:implementation}.

% TODO : cite their algorithmic paper?
% What makes STXXL good:
% http://stxxl.sourceforge.net/tags/master/introduction.html
% http://stxxl.sourceforge.net/tags/master/design.html
% http://stxxl.sourceforge.net/tags/master/design_algo_sorting.html
% http://stxxl.sourceforge.net/tags/master/citelist.html

% System
Our test machine was a server with a hyperthreaded quad-core 2.93 Ghz Intel Core i7-875K CPU and 16 GB RAM running
Ubuntu Server 14.04. Four Samsung 850 EVO 250GB SSDs were used for temporary storage for STXXL,
with a fifth identical drive used for temporary storage for SDSL-Lite and final graph output. In order to make
use of STXXL's parallel disk and asynchronous I/O support\footnote{\url{http://stxxl.sourceforge.net/tags/master/design_algo_sorting.html}},
the SSDs were not in a RAID configuration. The input files were read from a mechanical 2TB 7200 RPM disk.
% I might have been able to load the DSK files from a SSD thinking back... but it wouldn't make a difference except faster run time *for everything*

To minimize the effect of external factors on our results, each experiment was repeated three times with the minimum values
reported. The swap file was disabled, forcing the operating system to keep each graph completely in memory, and
there were no other users on the server.

% TODO : should construction be moved up and merged with the above?
% TODO : describe the construction algorithm
% TODO : describe why we needed external construction

\subsection{Test Data}

%To assess assembly quality, we aligned the reads to the {\em E.~coli} reference genome (substr.  K-12) using BWA (version 0.5.9) \cite{li2009fast} with default parameters.  We call a read {\em mapped} if BWA outputs an alignment for it and {\em unmapped} otherwise.  Analysis of the alignments revealed that 98\% of the reads mapped to the reference genome, representing an average depth of approximately $600\times$.  Next, we determined the amount of memory and time needed for our method for a larger dataset.  For that, we 
%The reference genome was also downloaded from the website (Reference genome GCA\_000001405.16).  
% Analysis of the alignments revealed that 98\% of the reads mapped to the reference genome, which represents 
% approximately 47x coverage of the genome.

In order to test the scalability of our approach, we repeated the experiment on readsets of varying size.
Our first data set consists of 27 million paired-end 100 character reads (strings)
from {\em E.~coli} (substr.  K-12). It was obtained from the NCBI Short Read Archive (accession 
ERA000206, EMBL-EBI Sequence Read Archive). The total size of this data set is around 2.3 GB compressed on disk (6 GB uncompressed).

The second data set is 36 million 155 character reads from the Human chromosome 14 Illumina reads used in the GAGE
benchmark\footnote{\url{http://gage.cbcb.umd.edu/}}, totalling 1.3 GB compressed on disk (6 GB uncompressed).

For our third data set we obtained 1,415 million  paired-end 100 character Human genome reads
(SRX01231) that were generated by Illumina Genome Analyzer (GA) IIx platform. The total size of this data set is 130 GB compressed on disk (470 GB uncompressed).

Our fourth data set is 700 million paired-end 101 character reads, and 131 paired-end 75 character reads from the short
insert libraries of the Parrot data (ERA201590) provided in Assemblathon 2\cite{assemblathon2}.
The total size of this data set is 64 GB compressed on disk (245 GB uncompressed).

%We ran DSK on each... The human data set required too much external memory, and the final dBG size would have been too large to fit in internal memory,
%so had to be re-run with the frequency threshold set to 2.

We used DSK~\cite{dsk} on each data set to find the unique $(K+1)$-mers.
It is usual to have DSK ignore low-frequency $(K+1)$-mers (as they may result from sequencing errors). 
However, removing such $(K+1)$-mers may result in the removal of some $k$-mers with $k \leq K$ that would 
otherwise have an acceptable frequency. We therefore set the frequency threshold to be as low as possible: $1$ (accepting 
all $(K+1)$-mers) for all data sets except for the Human genome data set, which was too big for our SSDs during construction,
and too big to fit into RAM afterwards. Hence, for the Human genome data set, the frequency threshold was $2$.

%\footnote{This increases the resulting size, but the size relative to the standard BOSS representation can still be compared.
%Variable-$K$ frequency filtering can be implemented using the $L^*$ vector, but will affect the dummy edge calculations, and
%is hence designated as future work.}.
% TODO: mention mercy kmers (Megahit) as another strategy? ultimately we just don't want gaps

A value of $K = 27$ was chosen
%\footnote{Note that the BOSS de Bruijn graph is defined in terms of $K+1$-mers, so DSK must be run for $K+1$ (i.e. 28 and 56) rather than for $K$.}
for the {\em E.~coli} data, and $K = 55$
for the Human data sets as these values produced good assemblies in previous papers (see, e.g.,~\cite{paul}). $K = 55$ was also chosen for the Parrot
data set, %as there was no clear choice for $K$ from the Assemblathon 2 contestants, and
as it produced a graph that almost filled the main memory.
The resulting file sizes and $(K+1)$-mer totals are shown in Table~\ref{tab:nav-time}.

% TODO: Run experiment with one data set over multiple Ks to see how it scales that way

% TODO: Time DSK (multithreaded) (I don't have timings for these)
%DSK took 28 and 58 minutes to run on the {\em E.~coli} and human data sets, respectively.

\subsection{Construction}
\label{sec:construction}

In order to convert the input DSK data to the format required by BOSS (in the correct order, with dummy edges, as required by both single-$K$ and variable-$K$ structures),
we use the following process, which has been designed with disk I/O in mind.

While reading the DSK input data, we generate and add the reverse complements for each $(K+1)$-mer, then sort them by their first $K$ symbols (the source nodes). Concurrently, we also sort another copy of the $(K+1)$-mers and their reverse complements by their last $K$ symbols (the target nodes). Let the resulting tables
be $A$ and $B$, respectively.

Next, we calculate the set differences $A-B$, comparing only the $K$-length prefixes to the $K$-length suffixes respectively. This tells us which source nodes do not
appear as target nodes, which we prepend with $\$$ signs to create the required incoming dummy edges ($K$ each), and then sort by the first $K$ symbols. Concurrently, we also calculate $B-A$ to give us
the nodes requiring outgoing dummy edges (to which we append $\$$). Let the resulting tables be $I$ and $O$, respectively. At this point $B$ can be deleted.
%\footnote{This will create duplicate strings, which can be avoided using Longest Common Prefix calculations.}.

Finally, we perform a three-way merge (by first $K$ symbols) of $A$, $I$, and $O$, outputting the rightmost column. In the case of the variable-$K$ graph,
we also calculate the $L^{*}$ values while merging. Finally, we construct the necessary succinct indexes from the output.

The time bottleneck in the above process is clearly in sorting the $A$ and $I$ tables. $|I|$ can be as big as $K|A|$, but in practice only $1\%$ or fewer
nodes require incoming dummies. Our elements are of size $\Oh{K}$, thus, overall, construction of both data structures takes $\Oh{K^2|A|\log|A|}$ time
and $\Oh{K^2|A|}$ space in theory, but in practice takes $\Oh{K|A|\log|A|}$ time and $\Oh{K|A|}$ space.

%The large file sizes necessitated an external construction scheme.

% Draw attention to the limited difference
%The only place that construction differs for the variable order de Bruijn graph is during the merge, where the longest common suffix length is calculated
%for consecutive edges, and written to disk. Finally, when constructing the rank and select structures, the variable order de Bruijn graph creates a Wavelet Tree
%as well.


\subsection{Results}

For each data set, the $(K+1)$-mers from DSK (and their reverse complements) were converted into the BOSS format
using the process outlined in \ref{sec:construction}, using the external memory vectors and multithreaded, external
memory sort from STXXL. The BOSS structure and $L^{*}$ wavelet tree were then built using indexes from SDSL-lite.

Construction times and structure sizes are shown in Table~\ref{tab:nav-time}.
While the variable-$K$ BOSS structure is around $30\%$ slower to build, and $2.6$ to $3.5$ times larger 
than the standard BOSS structure, this is clearly much faster and less space consuming than building 
$K$ separate instances of the BOSS structure. The peak RAM and disk usage is the same for both structures
except in the case of the Human genome data set, where the variable-$K$ BOSS structure used $22\%$ more RAM.
% TODO : repeat, if same results, trace RAM usage, work out why?

% identical peak disk and RAM
% TODO : Does this sounds week since we didn't measure building/storing each static-k dbg individually?
% TODO : Should we compare it to megahit?

%% TODO: fix the xs in this table? \times doesnt handle \em well though
%\begin{table}[h!]
%\begin{tabularx}{\textwidth}{@{\extracolsep{\fill} } r  c  c   c  c }
%						& \multicolumn{2}{c}{{\em Escherichia coli}} 		& \multicolumn{2}{c}{Human chromosome 14} \\
%						\cline{2-5}
%   						& BOSS 		& multi-K BOSS			& BOSS  		&  multi-K BOSS  \\
%\hline
%Wall Time (mins) & 19  & 25 {\em(1.32x)} & 153 & 203 {\em(1.33x)} \\
%Final Size (MB)  & 163 & 420 {\em(2.58x)} & 414 & 1416 {\em(3.42x)}\\
%Genome Size (bp) 	&  \multicolumn{2}{c}{4,639,221} 			&  \multicolumn{2}{c}{88,289,540} \\
%Number of Reads 			&  \multicolumn{2}{c}{27 M} 				&  \multicolumn{2}{c}{36.5 M}  \\
%DSK Time (mins) 	&  \multicolumn{2}{c}{28} 			&  \multicolumn{2}{c}{58} \\
%\hline
%\end{tabularx}
%\caption{Summary of data sets, as well as construction time and final space for BOSS de Bruijn graph and multi-$K$ de Bruijn graph. For the multi-$K$ BOSS representation, the
%increase factor is shown in parentheses.}
%\label{tab:build}
%\end{table}

%Due to the non-trivial engineering effort required to integrate a particular de Bruijn graph 
%structure in a live assembler, we compare our new data structure to the original BOSS structure
%by measuring average times for navigation operations.
To measure navigation functions $\forward$ and $\backward$ we took the mean time 
over 20,000 random queries. For the variable-$K$ graph, the $k$ values for each node 
were chosen randomly between $8$ and $K$. 
%(involves only a reverse lookup on a very small array). 
Results are shown in Table~\ref{tab:nav-time}. 
The new structure makes the $\forward$ operation $2.5$ to $3$ times slower for $k < K$, though we 
note that for $k = K$ $\forward$ time is identical.
The $\backward$ operation is much slower in the new structure, but is much less frequently 
used than $\forward$ in assembly algorithms (for a variation that supports fast
$\backward$ calculations, see \cite{varorder-latin}). We also measured $\lastchar$, which took only
nanoseconds on both structures.

To see how fast the order can be changed, we timed $\shorter$ and $\longer$ for
changes of $1$, $2$, $4$, and $8$ symbols. Our experiments show that in practice changing 
order by a single symbol ($\shorter_1$ and $\longer_1$) is a cheap operation, taking around the
same time as $\forward$. For larger changes in order, the time for $\shorter$ is stable
($\shorter_1$, $\shorter_2$, $\shorter_4$, and $\shorter_8$ all take roughly the same time),
whereas $\longer$ takes significantly more time as the difference in order increases. This is because
$\longer$ must compute a set of nodes, and the size of that set grows roughly exponentially with
the change in order ($\longer$ takes around $10 \mu$s per node when averaged over the size of the resulting set).
%If not every node in the result is to be visited, the set could instead be calculated lazily.

As expected, $\maxlen$ is very fast (it requires a single rank and select operation
over a bit vector), and only slightly affected when finding the specified outgoing edge label (which
uses a rank and select over the BOSS wavelet tree instead).

% TODO: add separate measurements for 2,4,8... but as order changes increase, the... factor is more evident.


