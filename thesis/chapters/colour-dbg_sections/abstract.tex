



%\begin{abstract}

%In the 20 years since it was introduced to bioinformatics by Idury and Waterman, the {\em de Bruijn graph} has become a mainstay of modern genomics, essential to genome assembly.
%The wide use and importance of de Bruijn graphs has led to a number of succinct representations, which aim to implement the graph in small space, while still supporting fast navigation operations.
  \textbf{Motivation:}
%  \noindent
  Iqbal et al. (Nature Genetics, 2012) introduced the {\em colored de Bruijn graph}, a variant of the classic de Bruijn graph, which is aimed at ``detecting and genotyping simple and complex genetic variants in an individual or population''.
Because they are intended to be applied to massive population level data, it is essential that the graphs be represented efficiently.
Unfortunately, current succinct de Bruijn graph representations are not directly applicable to the colored de Bruijn graph, which requires additional information to be succinctly encoded as well as support for non-standard traversal operations.\\
\textbf{Results:}
Our data structure dramatically reduces the amount of memory required to store and use the colored de Bruijn graph, with some penalty to runtime, allowing it to be applied in much larger and more ambitious sequence projects than was previously possible.\\
\textbf{Availability:} \url{https://github.com/cosmo-team/cosmo/tree/VARI}
%\end{abstract}
