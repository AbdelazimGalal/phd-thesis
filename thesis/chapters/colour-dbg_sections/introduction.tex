
\section{Introduction}

In the 20 years since it was introduced to bioinformatics by ~\cite{IW95}, the {\em de Bruijn graph} has become a mainstay of modern genomics, essential to genome assembly~\citep{how,sequel,ismb2015}. The near ubiquity of de Bruijn graphs has led to a number of succinct representations, which aim to implement the graph in small space, while still supporting fast navigation operations.  Formally, a de Bruijn graph constructed for a set of strings (e.g., sequence reads) has a distinct vertex $v$ for every unique $(k - 1)$-mer (substring of length $k - 1$) present in the strings, and a directed edge $(u, v)$ for every observed $k$-mer in the strings with $(k - 1)$-mer prefix $u$ and $(k - 1)$-mer suffix $v$. A contig corresponds to a non-branching path through this graph. See~\citep{how} for a more thorough explanation of de Bruijn graphs and their use in assembly. 

\cite{ICTFM12} introduced the {\em colored de Bruijn graph}, a variant of the classical structure, which is aimed at ``detecting and genotyping simple and complex genetic variants in an individual or population.'' The edge structure of the colored de Bruijn graph is the same as the classic structure, but now to each vertex ($(k - 1)$-mer) and edge ($k$-mer)
% FIXME: node coloring (CORTEX) looses information preserved in edge coloring(VARI), should we discuss this?  i.e. two nodes with the same color may or may not have a connecting edge with that color, but if you only color the nodes, you can't tell which is the case
is associated a list of colors corresponding to the samples in which the vertex or edge label exists. More specifically, given a set of $n$ samples, there exists a set $\mathcal{C}$ of $n$ colors $c_1, c_2, .., c_n$ where $c_i$ corresponds to sample $i$ and all $k$-mers and $(k-1)$-mers that are contained in sample $i$ are colored with $c_i$. A {\em bubble} in this graph corresponds to an undirected cycle, and is shown to be indicative of biological variation by \cite{ICTFM12}. 
{\sc Cortex}, the implementation of \cite{ICTFM12}, uses the colored de Bruijn graph to develop a method of assembling multiple genomes simultaneously, without losing track of the individuals from which $(k - 1)$-mers (and $k$-mers) originated. This graph is derived from either multiple reference genomes, multiple samples, or a combination of both.

Variant information of an individual or population can be deduced from structure present in the colored de Bruijn graph and the colors of each $k$-mer.
As implied by \cite{ICTFM12}, the ultimate intended use of colored de Bruijn graphs is to apply it to massive, population-level sequence data that is now abundant due to next generation sequencing technology (NGS) and multiplexing. These technologies have enabled production of sequence data for large populations, which has led to ambitious sequencing initiatives that aim to study genetic variation for agriculturally and bio-medically important species.  These initiatives include the {\em Genome 10K} project that aims to sequence the genomes of 10,000 vertebrate species~\citep{Haussler:2009}, the {\em iK5} project~\citep{Robinson:2011}, the 150 Tomato Genome ReSequencing project~\citep{tomato1,tomato2}, and the 1001 Arabidopsis project, a worldwide initiative to sequence cultivars of {\em Arabidopsis}~\citep{arabidopsis}.  Hence, the succinct colored de Bruijn graph is applicable in the context of these projects, in that it can assist in variation discovery within a species by analyzing all the data in these projects at once. 

In addition to species-specific initiatives, scientific and regulatory agencies are showing increased interest in shotgun metagenomic sequences for public health purposes~\citep{EMBL-EBI-Metagenomics,Miller2013}, specifically monitoring for antimicrobial resistance (AMR)~\cite{baquero_metagenomic_epi, port_2014_metagenomics_AMR_monitoring}.  AMR is considered one of the top public health threats, with fears that the spread of AMR will lead to increased morbitiy and mortality for many bacterial illnesses~\citep{CARB,FAOActionPlan2016}.  AMR occurs when bacteria express genetic elements that render them impervious to antibiotic treatments.  Importantly, these genetic resistance elements can be exchanged between distantly-related bacteria via multiple genetic mechanisms, which makes AMR an inherently population-level phenomenon~\citep{Baquero2013}.   Shotgun metagenomic sequencing allows access to the entire microbial population in a sample (the "metagenome"), which is of immense value for tracking and understanding the evolution of resistance elements within and across diverse bacteria\citep{MacLean2010}.  This metagenomics approach to AMR surveillance has been applied in both human and agricultural settings~\citep{noyes2016resistome,King2016}, generating hundreds of samples with terabytes of sequence data for relatively small studies.  Given the large number of samples and large size of sequence data involved in these whole-genome and metagenomic projects, it is imperative that the colored de Bruijn graph can be stored and traversed in a space- and time-efficient manner.
 
%the {\em Genome 10K} project that aims to sequence the genomes of 10,000 vertebrate species \cite{Haussler:2009}, the {\em iK5} project where the objective is to sequence the genomes of 5,000 arthropods \cite{Robinson:2011}, the 150 Tomato Genome ReSequencing project that aims to identify the sequence diversity within tomato \cite{tomato}, and the 1001 Arabidopsis  Project that is a worldwide initiative to sequence cultivars of Arabidopsis \cite{arabidopsis}. Given the large number of individuals and sequence data involved in these projects it is imperative that the colored de Bruijn graph is able to be stored and traversed in both a memory and time efficient manner.

\paragraph{Our Contribution}  
We develop an efficient data structure for storage and use of the colored de Bruijn graph. Compared to {\sc Cortex}, the implementation of \cite{ICTFM12}, our new data structure dramatically reduces the amount of memory required to store and use the colored de Bruijn graph, with some penalty to runtime. We demonstrate this reduction in memory through a comprehensive set of experiments across the following three datasets: (1)  four plant genomes, (2) 3,765 {\em Escherichia coli} assemblies,
 and (3) 87 sequenced metagenomic samples from commercial beef production facilities.  We show our method, which we refer to as $\ours$ (Finnish for color), has better peak memory usage on all these datasets. Our plant reference genomes dataset required 101 GB of RAM for  {\sc Cortex} to represent while $\ours$ required only 4 GB.  And  our
largest two datasets contain too many $k$-mers and colors for {\sc Cortex}'s data structure to represent in the 512 GB of RAM available on our bioinformatics servers. $\ours$ is a novel generalization of the succinct data structure for classical de Bruijn graphs due to \cite{BOSS12}, which is based on the Burrows-Wheeler transform of the sequence reads, and thus, has independent theoretical importance.

In addition to demonstrating the memory and runtime of $\ours$, we validate its output using the {\em E.coli} reference genome and a simulated variant.
%s 


\paragraph{Related Work} As noted above, maintenance and navigation of the de Bruijn graph is a space and time bottleneck in genome assembly. Space-efficient representations of de Bruijn graphs have thus been heavily researched in recent years. One of the first approaches was introduced by \cite{Simpson:2009} as part of the development of the ABySS assembler.  Their method stores the graph as a distributed hash table and thus requires 336 GB to store the graph corresponding to a set of reads from a human genome (>38x depth paired-end reads from Illumina Genome Analyzer II, HapMap: NA18507\footnote{\url{https://www.ncbi.nlm.nih.gov/sra/?term=SRA010896}}). 
 
 \cite{conway} reduced space requirements by using a sparse bitvector  (by \cite{bitvector}) to represent the $k$-mers (the edges), and used rank and select operations (to be described later) to traverse it. As a result, their representation took 32 GB for the same data set.  Minia, by \cite{wabi}, uses a Bloom filter to store edges. They traverse the graph by generating all possible outgoing edges at each node and testing their membership in the Bloom filter. Using this approach, the graph was reduced to 5.7 GB on the same dataset.  Contemporaneously, \cite{BOSS12} developed a different succinct data structure based on the Burrows-Wheeler transform~\citep{BW94} that requires 2.5 GB.  The data structure of \cite{BOSS12} is combined with ideas from IDBA-UD~\citep{idbaud} in a metagenomics assembler called MEGAHIT~\citep{megahit}.  In practice MEGAHIT requires more memory than competing methods  but produces significantly better assemblies.   \cite{paul} implemented the de Bruijn graph using an FM-index and {\em minimizers}.   Their method uses 1.5 GB on the same NA18507 data.  \cite{BFT} released the Bloom Filter Trie, which is another succinct data structure for the colored de Bruiin graph; however, we were unable to compare our method against it since  it only supports the building and loading of a colored de Bruijn graph and does not contain operations to support our experiments.  SplitMEM~\citep{splitmem} is a related algorithm to create a colored de Bruijn graph from a set of suffix trees representing the other genomes. Lastly, Lin et al. \citep{Lin} point out the similarity between the breakpoint graph, which is traditionally viewed as a data structure to detect breakpoints between genome rearrangements, and the colored de Bruijn graph. 
 

\paragraph{Roadmap} In the next section, we describe our succinct colored de Bruijn graph data structure, generalizing the stucture for classic de Bruijn graphs presented by ~\cite{BOSS12}. Section~\ref{sec:results} then elucidates the practical performance of the new data structure, comparing it to {\sc Cortex}. Section~\ref{sec:conclusion} offers some concluding remarks.

