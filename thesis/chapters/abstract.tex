%This thesis proposes succinct representations for the de Bruijn graph and
%different methods of analysis required to the traditional sanger sequenced data.
%The recent push towards ... sequencing yields longer reads at a significant cost in accuracy. So far, the new gen methods
%have not surpassed shotgun sequencing in practicality, but rather complement it.

% NOTE: I wanted to have something very to the point at the start
% The aim is absolutely to provide scalable representations of the most important de bruijn graph based outputs
% As well as the code to extend them even further
This thesis proposes the first Burrows--Wheeler Transform (BWT) based
\pyt{-'s changed to ---'s} succinct representation of the de Bruijn graph --- a data
structure used in most modern DNA assemblers.
The thesis also proposed a succinct representation of the variable-order de Bruijn graph --- the first formalisation of
a de Bruijn graph which can change order on the fly, and an alternative to the iterative de Bruijn graph,
which is used to improve accuracy in state-of-the-art metagenomic assemblers.
Finally, the thesis proposes a succinct representation of the coloured de Bruijn graph --- used for pangenomic
variant analyses and assembly.
These analyses are ultimately intended for massive population level short-read data, which is already being
generated at rates outpacing what current implementations can handle.
Hence it is paramount that these data structures are efficiently scalable.

The de Bruijn graph $G_K$ of a set of strings $S$ is a key data structure in short-read genome assembly,
which represents overlaps between all the $K$-length substrings ($K$-mers) of $S$. Construction and navigation of the
graph is a space and time bottleneck in practice, and the main hurdle for assembling large, eukaryote genomes.

The bottleneck is compounded by the fact that state-of-the-art assemblers do not build the
de Bruijn graph for a single order (value of $K$) but a so-called \textit{iterative} de Bruijn graph, for
multiple values of $K$. More precisely, they build or update $d$ de Bruijn graphs, each with a specific
order, i.e., $G_{K_1}$,  $G_{K_2}$, \ldots,  $G_{K_d}$. Although this paradigm increases the quality of the assembly produced,
it increases the run time by roughly a factor of $d$.

Furthermore, to detect genetic variants in populations, many pipelines require de Bruijn graphs
to be built for multiple genomes. Iqbal et al. (Nature Genetics 2012) introduced the \textit{colored} de
Bruijn graph, which exploits the similarity between organisms by storing them in a single graph, where
each $K$-mer has associated "color" information. That is, any $K$-mer can be mapped back to the subset of
string-sets for $C$ colors ${S_1, \ldots, S_C}$ that it appeared in.

%Because they are intended to be applied to massive single and population level data, it is essential that
%Bruijn graphs be represented efficiently.
% TODO: metagenomics, pangenomics

This thesis describes the first succinct de Bruijn graph representation based on the Burrows--Wheeler Transform
(a linchpin in cutting edge succint data structure research), which can be stored in $4m + o(m)$ bits for $m$
edges. This is much smaller than previous representations, while still enabling fast navigation operations:
indegree and outdegree of a node are computed in constant time, and outgoing and incoming edges with a given
label are computed in constant time and $\mathcal{O}(K)$ time, respectively. The data structure can be
constructed in $\mathcal{O}(K m \log{m})$ time.

Additionally, this thesis shows how to augment a succinct de Bruijn graph to support new operations which let
us change order on the fly, effectively representing all de Bruijn graphs up to some maximum order $K$ in a single
data structure, while only modestly increasing space usage, construction time, and navigation time compared
to a single order graph.

Finally, this thesis demonstrates how to store associated color data in a compressed format, dramatically reducing the amount
of memory required to store and use the colored de Bruijn graph, allowing it to be applied in much larger and more ambitious
sequence projects than was previously possible.

The code is open source, and is designed to flexibly support building new kinds of de Bruijn graphs (with any associated
edge and node data), as well as supporting use in the plethora of algorithms supplied by the Boost graph library. It makes
use of SDSL-lite and STXXL, the current best implementations of low-level succinct data structures and external memory
data structures respectively. It is available at: \url{http://github.com/cosmo-team/cosmo}.
% TODO: fix citations
% TODO: cite iterative
% TODO: external construction a contribution?
% TODO: mention flexible library
% TODO: add complexity for each graph
%We also show that it can be constructed on disk, reducing the barrier to building large dramatically.
%External construction.
%However, it is not without its limitations. Namely, even for a single genome, a graph can grow quite large. This necessitated
%At the mercy of the user-supplied k parameter, which affects the shape of the graph, and is sensitve to coverage.
%In theory, a high value of k is desired in order to capture repeat information, but in practice, due to read errors...
%In high coverage data, if a low value is selected the graph can become too tangled.
%In low coverage data, a high value of k can lead to broken paths.
% read the intros to our papers
%Complicating things, coverage is usually non-uniform within a sample. State of the art assemblers tackle this problem by
%making use of iterative de bruijn graphs, which repeatedly assemble the de bruijn graph for increasing k values, using each
%stages output to augment and correct the next stages input. This comes at the cost of...
%In this thesis, we introduce the variable order dBG, which is constructed once and allows the k parameter to be altered on the fly.
%This problem is especially evident in metagenomic analysis (where population samples are taken in e.g. soil)
