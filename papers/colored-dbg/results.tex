\section{Results}
\label{sec:results} 
We evaluated $\ours$ performance on three different datasets, described below.  For this evaluation, we compare peak memory, which was measured as the maximum resident set size, and CPU core time, measured as the user$+$system process time as our metrics.  In addition to evaluating performance, we also validated $\ours$ by the ability to correctly call bubbles known to be present in a simulated dataset.

Our software supports a variety of options.  It can consume $k$-mer counts from either Cortex's binary files or KMC2.  For all experiments, we use the KMC2 flow because using Cortex as a front end limits designs to only those that would fit in memory with Cortex.  Next, our software can compress the color matrix using either RRR or Elias-Fano encodings.  The SDSL-light implementation allows the color matrix to be compressed in an on-line fashion only using the Elias-Fano encoding.  This allows us to process larger designs, as the uncompressed matrix need never fit in RAM, and thus we use this option for all experiments.  Finally, STXXL (which holds temporary vectors during data structure construction) allows using internal or external memory.  Again, we used the more scalable external memory option for all experiments.
All experiments were performed on a machine with AMD Opteron model 6378 processors, having 512 GB of RAM and 64 cores. 

%and to accurately identify the origin of $k$-mers in a simulated metagenomics sample.  Finally, we present observations computed by $\ours$ on a collection of metagenomic samples taken from commercial beef production facilities.



\subsection{Datasets} \label{data}



The three different datasets were chosen in order to test and evaluate the performance of $\ours$ on a variety of diverse yet realistic data types that are likely to be used as input into $\ours$.  For the first two datasets which comprise single isolates, we use preassembled genomes.  Assembly serves to try correct sequencing errors which could otherwise falsely be detected as variants. To this end, {\sc Cortex} includes its own optional data cleaning operations.  However, by using instead the output of third party assembly software we can compare the colored de Bruijn graph performance on identical graphs.  Characteristics about these datasets are provided in Table \ref{tbl-datasets}.

%The first dataset contained six sub-strains of the {\em E. coli} K-12 strain reference genomes from NCBI.  The accession and substrains are 	
%    Each of the genomes contained approximately 4.6 million base pairs and had a median GC content of 49.9\%.


    \begin{table*}
      \small
   \centering
   \begin{tabular}{|c|l|c|c|}
     \hline
     {\bf Name} &  {\bf Accession Numbers} & {\bf Aprox. Size} & {\bf GC Content} \\

     \hline
     %% 6 {\it E. coli} & N/A &
     %% \pbox{4cm}{
     %% AP009048 -- W3110,  \\
	 %% CP009789 -- ER3413, \\
	 %% CP010441 -- ER3445, \\
	 %% CP010442 -- ER3466, \\
	 %% CP010445 -- ER3435, \\
	 %% U00096   -- MG1655 }&
     %% 4.6 Mbp & 49.9\% \\
     %% \hline
     
     \multirow{4}{*}{Plant Species }
     & Rice  (NC\_008394 to NC\_008405) & 430 Mbp & 43.42\% \\
     & Tomato  (NC\_015438 to NC\_015449) &  950 Mbp & 43.42\% \\
     & Corn  (NC\_024459 to NC\_024468) & 2.07 Gbp & 35.70\% \\
     & Arabidopsis  (NC\_003070 to NC\_003076) & 135 Mbp & 47.4\% \\
     \hline

     \emph{E. coli} strains &  N/A & \pbox{3cm}{avg=5.1 Mbp\\ min=2.9 Mbp\\ max= 7.7 Mbp} & 50.5\% \\

     \hline

     Beef safety &  PRJNA292471 & N/A & 44.3\% \\
     \hline
   \end{tabular}
       \caption{Characteristics of our datasets.  The \emph{E. coli} dataset represents 3,765 strains and hence only summary statistics for size and GC content are given. Accession numbers for this dataset as well as download procedure can be found in assembly\_summary.txt as discussed in the main text.}.
  \label{tbl-datasets}
   \end{table*}
% (\ref{tbl-ecoli}).



    % plants

    % VARI
    \begin{table*}

      \small
   \centering
   \begin{tabular}{|c|r|r|r|r|r|r|r|r|r|}

     \hline
      	\multicolumn{1}{|l}{}
   	& \multicolumn{2}{|c|}{{\sc Cortex}}	
	& \multicolumn{2}{|c|}{KMC2} 
	& \multicolumn{3}{|c|}{$\ours$-dBG} 
	& \multicolumn{2}{|c|}{$\ours$-C}  \\
        \hline
            {\bf Dataset} & {\bf CPU time} & {\bf Mem.} & {\bf CPU time} & {\bf Mem.} &{\bf CPU time} & {\bf Int. Mem.} & {\bf Ext. Mem.} & {\bf CPU time} & {\bf Mem.}  \\
            \hline
            Plants & 2h 25m 27s & 109,579 &  19m 50s & 4,335 & 1h 34m 37s & 5,388 & 156,504 & 3m 09s & 3,528 \\
%% (py3.5env)muggli@oak:/s/oak/b/nobackup/muggli/abdoserver2/plants$ python ../appraise.py arabidopsis/kmc.stdout
%% ut 15.1 st 1.97 tot 17.07 maxrss 376012
%% (py3.5env)muggli@oak:/s/oak/b/nobackup/muggli/abdoserver2/plants$ python ../appraise.py arabidopsis/kmc_sort.stdout
%% ut 48.04 st 3.55 tot 51.589999999999996 maxrss 1127104
%% (py3.5env)muggli@oak:/s/oak/b/nobackup/muggli/abdoserver2/plants$ python ../appraise.py corn/kmc.stdout
%% ut 247.65 st 19.19 tot 266.84000000000003 maxrss 4334804
%% (py3.5env)muggli@oak:/s/oak/b/nobackup/muggli/abdoserver2/plants$ python ../appraise.py corn/kmc_sort.stdout .stdout
%% ut 342.36 st 24.67 tot 367.03000000000003 maxrss 2087636
%% (py3.5env)muggli@oak:/s/oak/b/nobackup/muggli/abdoserver2/plants$ python ../appraise.py rice/kmc.stdout
%% ut 20.5 st 2.46 tot 22.96 maxrss 399276
%% (py3.5env)muggli@oak:/s/oak/b/nobackup/muggli/abdoserver2/plants$ python ../appraise.py rice/kmc_sort.stdout
%% ut 56.58 st 4.35 tot 60.93 maxrss 1359016
%% (py3.5env)muggli@oak:/s/oak/b/nobackup/muggli/abdoserver2/plants$ python ../appraise.py tomato/kmc.stdout
%% ut 101.55 st 10.88 tot 112.42999999999999 maxrss 2344340
%% (py3.5env)muggli@oak:/s/oak/b/nobackup/muggli/abdoserver2/plants$ python ../appraise.py tomato/kmc_sort.stdout
%% ut 269.75 st 21.42 tot 291.17 maxrss 2044520    
    
%% (py3.5env)muggli@oak:/s/oak/b/nobackup/muggli/abdoserver2/plants/vari2$ python ../../appraise.py cosmo-build-numcols4.stdout
%% ut 5428.25 st 249.03 tot 5677.28 maxrss 5387872
%% (py3.5env)muggli@oak:/s/oak/b/nobackup/muggli/abdoserver2/plants/vari2$ python ../../appraise.py pack-color.stdout
%% ut 177.22 st 12.66 tot 189.88 maxrss 2200508
%% (py3.5env)muggli@oak:/s/oak/b/nobackup/muggli/abdoserver2/plants/vari2$ python ../../appraise.py cosmo-color.stdout
%% ut 117514.62 st 46.88 tot 117561.5 maxrss 3527548

%% % cortex
%% (py3.5env)muggli@oak:/s/oak/b/nobackup/muggli/abdoserver2/plants$ python ../appraise.py arabidopsis/cortex.stdout
%% ut 176.3 st 562.25 tot 738.55 maxrss 54515904
%% (py3.5env)muggli@oak:/s/oak/b/nobackup/muggli/abdoserver2/plants$ python ../appraise.py corn/cortex.stdout
%% ut 1677.18 st 2106.7 tot 3783.88 maxrss 54527916
%% (py3.5env)muggli@oak:/s/oak/b/nobackup/muggli/abdoserver2/plants$ python ../appraise.py rice/cortex.stdout
%% ut 223.76 st 695.04 tot 918.8 maxrss 54525120
%% (py3.5env)muggli@oak:/s/oak/b/nobackup/muggli/abdoserver2/plants$ python ../appraise.py tomato/cortex.stdout
%% ut 518.64 st 1068.51 tot 1587.15 maxrss 54528304
%% (py3.5env)muggli@oak:/s/oak/b/nobackup/muggli/abdoserver2/plants/cortex$ python ../../appraise.py combine.stdout
%% ut 1209.48 st 489.85 tot 1699.33 maxrss 109579316
%% (py3.5env)muggli@oak:/s/oak/b/nobackup/muggli/abdoserver2/plants/cortex$ python ../../appraise.py bubbles_all4.stdout
%% ut 4264.09 st 7163.3 tot 11427.39 maxrss 109582008

%% % E. coli     (vari only)
    
    \emph{E. coli} ($k$=32)         & N/A & N/A & 3h 15m 40s & 104 & 9h 30m 11s & 126,777 & 319,328 &  53m 54s & 42,043 \\


%% (py3.5env)muggli@oak:/s/oak/b/nobackup/muggli/abdoserver2/ncbi_ecoli/logs$ python ../../appraise.py k32kmc2.stdout
%% ut 3197.4000000000397 st 1036.0099999999993 tot 4233.410000000039 maxrss 55592
%% (py3.5env)muggli@oak:/s/oak/b/nobackup/muggli/abdoserver2/ncbi_ecoli/logs$ python ../../appraise.py k32sort.stdout
%% ut 6444.450000000013 st 1062.9999999999732 tot 7507.449999999986 maxrss 103836
%% (py3.5env)muggli@oak:/s/oak/b/nobackup/muggli/abdoserver2/ncbi_ecoli/logs$ python ../../appraise.py ../k32vari/cosmo-build5.stdout
%% ut 32104.24 st 2106.82 tot 34211.060000000005 maxrss 126776588
%% (py3.5env)muggli@oak:/s/oak/b/nobackup/muggli/abdoserver2/ncbi_ecoli$ python ../appraise.py k32vari/pack-color.stdout
%% ut 2904.33 st 330.5 tot 3234.83 maxrss 42042524
%% (py3.5env)muggli@oak:/s/oak/b/nobackup/muggli/abdoserver2/ncbi_ecoli$ python ../appraise.py k32vari/bubbles_nostartstring_bincol.stdout
%% ut 13889.3 st 340.89 tot 14230.189999999999 maxrss 42172540

    \emph{E. coli} ($k$=48)         & N/A & N/A & 4h 35m 29s & 149 & 10h 47m 46s & 128,077 & 427,460 & 1h 02m 07s & 42,100 \\

%% (py3.5env)muggli@oak:/s/oak/b/nobackup/muggli/abdoserver2/ncbi_ecoli/logs$ python ../../appraise.py k48kmc2.stdout
%% ut 5323.469999999971 st 1183.0299999999845 tot 6506.499999999955 maxrss 51568
%% (py3.5env)muggli@oak:/s/oak/b/nobackup/muggli/abdoserver2/ncbi_ecoli/logs$ python ../../appraise.py k48sort.stdout
%% ut 8424.160000000009 st 1599.590000000002 tot 10023.750000000011 maxrss 149000
%% (py3.5env)muggli@oak:/s/oak/b/nobackup/muggli/abdoserver2/ncbi_ecoli/logs$ python ../../appraise.py ../k48vari/cosmo-build5.stdout
%% ut 36599.0 st 2267.36 tot 38866.36 maxrss 128076768
%% (py3.5env)muggli@oak:/s/oak/b/nobackup/muggli/abdoserver2/ncbi_ecoli$ python ../appraise.py k48vari/pack-color.stdout
%% ut 3386.1 st 341.69 tot 3727.79 maxrss 42100288
%% (py3.5env)muggli@oak:/s/oak/b/nobackup/muggli/abdoserver2/ncbi_ecoli$ python ../appraise.py k48vari/bubbles_nostartstring_bincol.stdout
%% ut 16309.5 st 362.26 tot 16671.76 maxrss 42262504

    \emph{E. coli} ($k$=64)         & N/A & N/A & 5h 05m 27s & 189 & 11h 21m 08s & 127,523 & 522,576 & 1h 09m 07s & 42,134 \\
%% (py3.5env)muggli@oak:/s/oak/b/nobackup/muggli/abdoserver2/ncbi_ecoli/logs$ python ../../appraise.py k64kmc2.stdout
%% ut 6119.789999999991 st 1447.560000000001 tot 7567.349999999992 maxrss 72288
%% (py3.5env)muggli@oak:/s/oak/b/nobackup/muggli/abdoserver2/ncbi_ecoli/logs$ python ../../appraise.py k64sort.stdout
%% ut 8868.87999999998 st 1891.4600000000205 tot 10760.34 maxrss 189240
%% (py3.5env)muggli@oak:/s/oak/b/nobackup/muggli/abdoserver2/ncbi_ecoli/logs$ python ../../appraise.py ../k64vari/cosmo-build2.stdout
%% ut 38531.56 st 2336.71 tot 40868.27 maxrss 127522776
%% (py3.5env)muggli@oak:/s/oak/b/nobackup/muggli/abdoserver2/ncbi_ecoli$ python ../appraise.py k64vari/pack-color.stdout
%% ut 3710.34 st 436.79 tot 4147.13 maxrss 42134080
%% (py3.5env)muggli@oak:/s/oak/b/nobackup/muggli/abdoserver2/ncbi_ecoli$ python ../appraise.py k64vari/bubbles_nostartstring_bincol.stdout
%% ut 19075.25 st 611.96 tot 19687.21 maxrss 42324840

    Beef safety & N/A & N/A & 34h 04m 46s & 11,688 & 82h 42m 48s & 109,091 & 4,378,840 & 6h 44m 12s & 217,705 \\
%% % beef safety (vari only)

%% (py3.5env)muggli@oak:/s/oak/b/nobackup/muggli/abdoserver2/eightyeight$ python ../appraise.py kmc_count.log
%% ut 61046.46000000002 st 20299.359999999982 tot 81345.82 maxrss 11688072
%% (py3.5env)muggli@oak:/s/oak/b/nobackup/muggli/abdoserver2/eightyeight$ python ../appraise.py union.log
%% ut 32831.659999999996 st 8510.230000000001 tot 41341.89 maxrss 5594060
%% (py3.5env)muggli@oak:/s/oak/b/nobackup/muggli/abdoserver2/eightyeight/vari2$ python ../../appraise.py cosmo_build_kmc_files_orderby_location_then_matrix_corrected2.stdout
%% ut 266381.96 st 31386.87 tot 297768.83 maxrss 109090672
%% (py3.5env)muggli@oak:/s/oak/b/nobackup/muggli/abdoserver2/eightyeight/vari2$ python ../../appraise.py pack-color-88.stdout
%% ut 21204.29 st 3048.3 tot 24252.59 maxrss 217704948
%% (py3.5env)muggli@oak:/s/oak/b/nobackup/muggli/abdoserver2/eightyeight/vari2$ python ../../appraise.py cosmo_color_pd_walkrefs_corrected
    %% ut 296.88 st 932.91 tot 1229.79 maxrss 245545288
    \hline
   \end{tabular}
       \caption{Data structure construction performance measurements.  CPU time is user plus system time as reported by `/bin/time'.  (Internal) memory is reported in megabytes and is the maximum resident set size. KMC2 includes both counting and sorting $k$-mers. $\ours$-dBG forms the $k$-mer union and builds the succinct de Bruijn graph. $\ours$-C compresses the color matrix.}
  \label{tbl-buildper}

      \end{table*}





    

    

    Our first performance dataset comprises reference genomes for four different plant species:
    \emph{Oryza sativa Japonica} (rice)\footnote{\url{http://rice.plantbiology.msu.edu/annotation_pseudo_current.shtml}}\citep{rice},
    \emph{Solanum lycopersicum} (tomato)\footnote{\url{ftp://ftp.solgenomics.net/tomato_genome/assembly/build_2.50/SL2.50ch00.fa.tar.gz}}\citep{tomato1,tomato2},
    \emph{Zea mays} (corn)\footnote{\url{ftp://ftp.ncbi.nlm.nih.gov/genomes/all/GCF/000/005/005/GCF_000005005.1_B73_RefGen_v3/GCF_000005005.1_B73_RefGen_v3_genomic.fna.gz}}\citep{corn}, and
    \emph{Arabidopsis thaliana} (Arabidopsis)\footnote{\url{ftp://ftp.ensemblgenomes.org/pub/plants/release-34/fasta/arabidopsis_thaliana/dna/}}\citep{swarbreck}.
    This represents a sufficiently large dataset for comparing the performance of $\ours$ with {\sc Cortex}.  

%The genome sizes and GC content were 430 Mbp and 43.42\% \citep{rice}, 950 Mbp and 43.42\% \citep{tomato1,tomato2}, 2.07 Gbp and 35.70\% \citep{corn}, and 135 Mbp and 47.4\% \citep{swarbreck}, respectively

Our second performance dataset consists of the set of all 3,765  NCBI GenBank assemblies\footnote{\url{ftp://ftp.ncbi.nlm.nih.gov/genomes/genbank/bacteria/assembly_summary.txt}}\footnote{\url{https://www.ncbi.nlm.nih.gov/genome/doc/ftpfaq/}} having the organism\_name field equal to ``Escherichia coli'' as of March 22, 2016.  To evaluate the effects of varying $k$-mer size, we ran this dataset with $k={32,48,64}$.  The union of all assemblies contains 158,501,209 $k$-mers for $k$=32, 205,938,139 $k$-mers for $k$=48, and 251,764,413 $k$-mers for $k$=64.  The minimum, maximum, and average assembly lengths are 2,911,360 bp, 7,687,202 bp, and 5,156,744 bp, respectively.



Our third performance dataset consists of 87 metagenomic samples\footnote{\url{https://www.ncbi.nlm.nih.gov/bioproject/292471}} taken at various timepoints during the beef production process from eight pens of cattle in two beef production facilities by \cite{noyes2016resistome}.  Sequentially, these timepoints were feedlot arrival, feedlot exit, slaughter transport truck, slaughter holding, and slaughter trimmings and sponges.  Sample reads were preprocessed using trimmomatic v0.36 by Bolger {\it et al.}~\citep{bolger2014trimmomatic}.  Although further assembly or error correction would have been possible, it would reduce the biological variation which may be useful for some queries.  Furthermore, building the data structure on uncorrected data better stresses our representation method.  Samples were then arranged into groups based on the sample timepoints. The original study used these samples to demonstrate the advantages of shotgun metagenomic sequencing in tracking the evolution of antimicrobial resistance longitudinally within a complex environment like beef production; the results suggested that selective pressures occurred within the feedlot, but that slaughter safety measures drastically reduced both bacterial and AMR levels.  In addition to the metagenomic samples, we included 4,062 AMR genes from the previously mentioned gene databases\footnote{\url{https://meg.colostate.edu/MEGaRes/}}.  23 genes in the databases containing IUPAC codes other than the four bases were filtered out as KMC2 and the succinct de Bruijn graph were configured with a four symbol alphabet.  Because we have the reference to guide the traversal, all AMR genes were combined into a single color.  By combining AMR genes, the uncompressed color matrix that exists on disk during sorting and as intermediate file is much smaller (still occupying 1.2 TB), thus accelerating the permutation during construction and reducing the external memory and disk space requirements.  The union of all samples and genes contains 40,995,794,366 32-mers and the GC content is 44.3\%.  While our server has enough RAM to represent a dataset with twice the memory footprint, this dataset nearly exhuasted the approximately 10 TB of disk space available when intermediate files were preserved.  Thus this dataset is on the order of the upper limit for $\ours$ in practice.

% TODO 
% the reference genome for E coli K-12 subst. MG 1655
% make clear that has two genomes
% a copy with some simulated variations inserted
Finally, for validation purposes, we generated a dataset\footnote{\url{https://github.com/cosmo-team/cosmo/tree/VARI/experiments/ecoli_validation}} comprising two genomes: (1) \emph{E. coli} K-12 substraing MG 1655 reference genome, and (2) a copy of the reference genome to which we applied various simulated mutations.  We simulated mutations by choosing 100 random loci and either inserting, deleting, or replacing a region of random length ranging from 200-500 bp.  For each mutation locus, we record the flanking regions and the two variants (original reference and simulated) as a ground truth bubble.  



% software configuration

\subsection{Time and Memory Usage}

% what is bubble calling
To measure $\ours$'s resource use and compare with {\sc Cortex} by \cite{ICTFM12} where possible, we constructed the colored de Bruijn graph for the plant dataset, the \emph{E. coli} assembly dataset and the beef safety dataset. Construction time and memory is detailed in Table \ref{tbl-buildper}. We performed {\em bubble calling} on the first two and recorded peak memory usage and runtime.  Direct resource comparison with {\sc Cortex} was only possible on the smallest  dataset, as the largest two have too many $k$-mers and colors to fit in memory on our machine with {\sc Cortex}.  Based on the data structure defined in {\sc Cortex}'s source as well as the supplementry information provided by Iqbal {\it et al.}, it would have required more than 3 TB of RAM and more than 18 TB of RAM for its hash table entries alone, respectively.




% construction statistics
%For the {\it E. coli} assembly dataset, after $k$-mer counting with KMC2, construction took 16 hours and 387 GB of RAM for this dataset. For the beef safety samples, it took KMC2 34 hours and 12 GB of RAM to $k$-mer count the 887 GB of trimmed reads. $\ours$ required 87 hours, 218 GB of RAM, and 4.4 TB of external memory to build the succinct colored de Bruijn graph.  The uncompressed $C$ temporary file was 1.4 TB in size, which compressed to 196 GB with Elias-Fano encoding.



%For the simulated metagenomic dataset, each AMR gene was assigned a separate color allowing the match-color algorithm to be applied.  Thus no graph traversal times are reported although the size of the structures are reported.  
%We ran the bubble calling algorithm between the colors for the assemblies ``GCF\_000005845.2\_ASM584v2\_genomic'' and ``GCF\_000006665.1\_ASM666v1\_genomic''.

%% On the final two datasets, the simulated metagenomic sample and beef production samples, we are interested in detecting the presence of known AMR genes among the reads.  
%% we explore the feasibility of using a colored de Bruijn graph for AMR gene detection among metagenomic samples.
%% We do so by visiting $k$-mers found in AMR genes and look for those that co-occur in the metagenomic samples.

%% % discussion of k-mer intersection
%% On the final two datasets, instead of using either of {\sc Cortex}'s bubble calling algorithms, which are designed for variant detection we are interested in presence of the AMR genes in the sample.  Variants may represent read errors or largely homologous genes missing the antibiotic resistant determinant. in samples taken from a pair of individuals or individual and reference,   In this applications, instead of looking for variants, we are looking for similarity
%% % varants -- regions that differ flanked by homologous regions, we are looking strictly for homologous regions -- as the metagenomic sample may be plagued with incomplete coverage
%% in the presence of incomplete coverage, a mix of many individuals, thus weakening common read error correction assumptions, and significant homology among AMR genes and their possibly co-occuring ancestral variants leading to more tangled graphs which impair {\sc Cortex}'s cleaned individual oriented algorithm.  In the absence of a colored, metagenomic-aware traversal algorithm, we focus only on regions of similarity and allow unlimited sized gaps between them, where the metagenomic sample color may be highly tangled or unconnected.  

%% One thing that distinguishes such datasets from pangenomic datasets, such as those targeted by the bloom filter trie, is that these datasets may have significantly smaller amounts of redundancy.  On the beef safety dataset, for example, two thirds of $k$-mers occur with multiplicity of one across the whole collection.  

%% On the simulated metagenomic dataset, 

%% For the $k$-mer set operation was not necessary and no time comparison is reported.

%% Hence this experiment demonstrates the viability of AMR gene detection via $k$-mer set operations.

%% For the 87 metagenomic sample dataset, the full data structures were loaded in memory, however only the length of each gene sequence need be traversed, resulting in the relatively fast 21 minute load and traverse time compared with bubble calling across a genome.


%% On the 87 beef safety sample, we implemented a variant of {\sc Cortex}'s path divergence algorithm, which locates bubbles where one arm of the bubble is allowed to have branches because its path is guided by a reference sequence.  In our version, we traverese the walk specified by each AMR gene sequence and count the number of edges shared with all samples from each sample location in the beef production pipeline.  


%% We note the benefit  of using a rank-select capable dictionary datastructure such as RRR or Elias-Fano encoded bit vector and storing the color matrix in linearized row major order.  Samples were arranged such that all samples taken from the same location within the production process were placed in consecutive columns in $C$.
%% %To analyze colored de Bruijn graphs with {\sc Cortex}'s algorithms, we must do all pairs comparisons or at least compare every sample of interest against some reference color; both of which become a growing runtime burden as the number of colors increases.
%% In this configuration, our data structure allows us to work on groups of samples at a time while still maintaining each sample individuality for analyses when that is of interest.  Specifically, if colors are ordered by relevant groups, such as by  the location samples were taken in the beef production pipeline.  With such grouping, we can count the number of sample colors from a group that are present in a given edge quickly.  This is computed in constant time as the difference between rank queries on the group's column interval bounds.

In order to test query performance characteristics, various experiments were performed on all three performance datasets described in the previous subsection.  Datasets varied in the number of $k$-mers in the graph from 158 million to over 40 billion, the number of colors, from 4 to 3,765, and degree of homology from disperate plants to the single \emph{E. coli} species.  This diversity shows the space savings achievable when the population is largely homologous, as is the case with the \emph{E. coli} dataset, where the graph component is relatively small, in contrast to the plant dataset, where the graph component is relatively large. As can be seen in Table \ref{tbl-cosmo}, where directly comparable, $\ours$ used an order of magnitude less than the peak memory that {\sc Cortex}  required but required greater running time.  This memory and time trade-off is important in larger population level data.  %Given that {\sc Cortex} requires 100.93 GB of space for four plant species, it would be perceptibly infeasible to run it on the i5K initiative dataset that contains the genetic sequence data for 5,000 insect species.
This is highlighted by our largest two datasets which could not be run with {\sc Cortex}.
Hence, lowering the memory usage in exchange for higher running time deserves merit in contexts where there is data from large populations. 
 
%We note that the ratios were greater for the AMR dataset, which was likely due to the greater number of colors in the AMR versus the \emph{E. coli} and plant datasets (55 versus 6 and 4, respectively).


\begin{table*}%[h!]
  \small
  \centering
  \begin{tabular}{| l | r | r | c | c | c |c |}
   	\hline
	\multicolumn{1}{|l}{}
   	& \multicolumn{1}{r}{}	
	& \multicolumn{1}{r}{} 
	& \multicolumn{2}{|c|}{{\sc Cortex}} 
	& \multicolumn{2}{|c|}{$\ours$}  \\
	\hline
	 Dataset & No. of $k$-mers & Colors & Memory & Time & Memory & Time \\
	\hline
%	\emph{E. coli} reference genomes ($k$=32)			& 4,628,261 		& 6 	& 363.64 MB 	& 9 s 	& 72.38 MB  	& 1m 19s\\
	Plants	($k$=32)				& 1,709,427,823 	& 4 	& 100.93 GB 	& 2h 18m	& 3.53 GB (sdBG=0.89 GB, sC=1.95 GB) 	& 32h 39m \\
     \emph{E. coli}  ($k$=32)         & 158,501,209       & 3,765 & N/A        & N/A      &  42.17 GB (sdBG=0.09 GB, sC=38.35 GB)     & 3h 57m  \\
     \emph{E. coli}  ($k$=48)         & 205,938,139       & 3,765 & N/A        & N/A      &  42.26 GB (sdBG=0.11 GB, sC=38.42 GB)     & 4h 38m  \\
     \emph{E. coli}  ($k$=64)         & 251,764,413       & 3,765 & N/A        & N/A      &  42.32 GB (sdBG=0.13 GB, sC=38.45 GB)     & 5h 28m  \\
    Beef safety ($k$=32)                            & 40,995,794,366    & 88    & N/A        & N/A   & 245.54 GB (sdBG=27.08 GB, sC=200.34 GB)     & N/A \\
 	\hline
	\end{tabular}
  \caption{Comparison between the peak memory and time usage required to store all the $k$-mers and run bubble calling on the data in {\sc Cortex} and $\ours$.
    %$k= 31$ was used for all datasets.
    The peak memory is given in megabytes (MB) or gigabytes (GB). The running time is reported in seconds (s), minutes (m), and hours (h).  The succinct de Bruijn graph and compressed color matrix components of the memory footprint are listed in parenthesis as sdBG and sC, respectively.}
 \label{tbl-cosmo}
\end{table*}


\subsection{\protect{Validation on Simulated \emph{E. coli}}}

We ran the implementations of bubble calling from both $\ours$ and {\sc Cortex}, using $k$=32 on the simulated \emph{E. coli} dataset.  Both tools reported the same set of 223 bubbles, 55 of which were in the ground truth set.  This ensures our software faithfully implements the original data handling capabilities of {\sc Cortex}.  For biological implications of colored de Bruijn graph variant calls and in particular with parameter choices such as $k$ see Iqbal {\it et al.} \citep{ICTFM12}.
%Less than perfect accuracy is expected for a de Bruijn graph based variant caller given that genomic repeats 1) introduce tangles which confound detection of bubble arms leading to false negatives and 2) homologous regions in disperate loci may still match the criteria for a bubble leading to false positives (i.e. bubbles may form in the graph which are not the consequence of a mutation event at a single locus).  These drawbacks are not particular to $\ours$, and are intrinsic to any method consuming shotgun read data.  

\subsection{Observations on Beef Safety Dataset}

%While the primary purpose of this experiment is to measure the performance in terms of memory footprint, we can examine the data we collected during traversal which may be useful to biologists and spawn hypotheses for further investigation.
While the beef safety dataset was primarily used for measuring the scalability of $\ours$ and to determine if representing a dataset of this type and size was possible, we used $\ours$ to additionally make observations about the presence of AMR genes in the beef production dataset.  As previously described, during our path divergence derived algorithm, we compute a count of how many $k$-mers in each AMR gene are found across all samples within a sample group.  This algorithm need only traverse the AMR genes, so despite the size of the overall dataset, it only took 20 minutes to load and access the necessary parts of the data structure.  In contrast, if bubble calling were to run at the same rate for this dataset as for the {\it E. coli} assembly dataset, it would take 3,001 hours to complete, thus suggesting value in a targeted inquiry approach on datasets of this size.  

Since longer genes have more $k$-mers, the counts are likely to be larger, as are those from larger sample groups.  To make these counts comparable, we normalize by both gene length and sample group size.  We can then examine the number of genes having a disproportionately large ($>$ 3 std. dev. above mean) shared $k$-mer count for each gene and sample group combination.  The number of such genes with disproportionately large normalized counts in each sample group were:  feedlot arrival - 304, feedlot exit - 93, transport truck - 230, slaughter holding - 16, and slaughter trimmings and sponges - 0.
%Despite biases in the sampling process that need to be taken into account for substantial biological implications, one validating property of these observations is that there were no AMR genes with disproporitionately large representation in the samples taken on exit from the slaughter location.
This observation supports the conclusion of \cite{noyes2016resistome}, namely, that antimicrobial interventions during slaughter were effective in reducing AMR gene presence in the trimmings and sponge samples, which represent the finished beef products just before they are shipped to retail outlets for human consumption. 
