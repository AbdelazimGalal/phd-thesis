\section{Conclusion}\label{sec:conclusion}

We have described a method for efficiently representing multiple de Bruijn graphs 
of different orders in a single succinct data structure. As well as the usual graph 
traversal operations, the data structure supports new operations which allow the order 
of the de Bruijn graph to be changed on the fly. This data structure has the potential 
to greatly improve the memory and space usage of current state-of-the-art assemblers 
that build the de Bruijn graph for multiple values of $K$, and ultimately allow those
assemblers to scale to large, eukaryote genomes. The integration of our new data 
structure into a real assembler is thus our most pressing avenue for future work.
%as well as further reducing the size by developing a method for variable-order
%frequency filtering.
