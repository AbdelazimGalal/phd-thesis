
\section{Background}\label{sec:related}

\paragraph{Rank and Select}
\label{sec:rank}
Two basic operations
used in almost every succinct and compressed data structure are {\em rank} and
{\em select}. Given a sequence (string) $S[1,n]$ over an alphabet $\Sigma =
\{1,\ldots,\sigma\}$, a character $c \in \Sigma $, and integers
$i$,$j$, $\rank_c(S,i)$ is the number of times that $c$ appears in
$S[1,i]$, and $\select_c(S,j)$ is the position of the $j$-th
occurrence of $c$ in $S$.
%There is a great variety of techniques to answer these queries,
%with suitability depending on the nature of the sequence, for example, on
%whether or not it will be compressed and on the size of the alphabet.
For a binary string $B[1,n]$, the classic solution for rank and select~\cite{Mun96} 
is built upon the input sequence, requiring $o(n)$ additional bits.
Generally, $\rank_1$ and $\select_1$ are considered the default
rank and select queries.
More advanced solutions (e.g.~\cite{bitvector}) achieve zero-order 
compression of $B$,
%For example, the several structures (e.g.~\cite{bitvector}),
%(see also~\cite{kkp2014}), 
representing it in just $nH_0(B) + o(n)$ bits of space, and supporting $\rank$ and
$\select$ operations in constant time. 
%Several practical implementations
%and improvements of RRR exists (see, e.g.,~\cite{kkp2014}).

\paragraph{Wavelet Trees}
\label{sec:WVT}
To support rank and select on larger alphabet strings, the wavelet tree~\cite{ggv2003,n2013} is a 
commonly used data structure that occupies $n\log\sigma + o(n\log\sigma)$
bits of space and supports $\rank$ and $\select$ queries in $\Oh{\log\sigma}$ time.
Wavelet trees also support a variety of more complex queries on the underlying string (see, e.g.~\cite{gnp2012}),
in $\Oh{\log\sigma}$ time, and we will make use of some of this functionality in Section~\ref{sec:implementing}.

\paragraph{Coloured de Bruijn Graphs}
\textbf{This would probaby go better in the introduction but it was being edited.}
SplitMEM in 2014, by S. Marcus. et. al.~\cite{splitmem}. Sequence Bloom Trees in 2015, by B. Solomon and C. Kingsford~\cite{SBT}.
Later, Bloom Filter Trie in 2015, by G. Holley et. al.~\cite{BFT}.
