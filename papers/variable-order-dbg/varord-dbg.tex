\documentclass[10pt,journal,compsoc]{IEEEtran}


\newtheorem{lemma}{Lemma}
\newtheorem{proposition}{Proposition}
\newtheorem{definition}{Definition}
\newtheorem{observation}{Observation}
\newtheorem{claim}{Claim}
\newtheorem{theorem}{Theorem}
\newcommand{\proof}[1]{{\em Proof}}

\newcommand{\Oh}[1]
  {\ensuremath{\mathcal{O}\!\left( {#1} \right)}}
\newcommand{\forward}
  {\ensuremath{\mathsf{forward}}}
\newcommand{\backward}
  {\ensuremath{\mathsf{backward}}}
\newcommand{\lastchar}
  {\ensuremath{\mathsf{lastchar}}}
\newcommand{\shorter}
  {\ensuremath{\mathsf{shorter}}}
\newcommand{\longer}
  {\ensuremath{\mathsf{longer}}}
\newcommand{\maxlen}
  {\ensuremath{\mathsf{maxlen}}}
\newcommand{\rank}
  {\ensuremath{\mathsf{rank}}}
\newcommand{\select}
  {\ensuremath{\mathsf{select}}}
\newcommand{\rsucc}
  {\ensuremath{\mathsf{succ}}}
\newcommand{\nodelabel}
  {\ensuremath{\mathsf{label}}}


\newcommand{\ignore}[1]{}

% *** CITATION PACKAGES ***
\ifCLASSOPTIONcompsoc
  % IEEE Computer Society needs nocompress option
  % requires cite.sty v4.0 or later (November 2003)
  \usepackage[nocompress]{cite}
\else
  % normal IEEE
  \usepackage{cite}
\fi

% *** GRAPHICS RELATED PACKAGES ***
\ifCLASSINFOpdf
  \usepackage[pdftex]{graphicx}
  % declare the path(s) where your graphic files are
  % \graphicspath{{../pdf/}{../jpeg/}}
  % and their extensions so you won't have to specify these with
  % every instance of \includegraphics
  \DeclareGraphicsExtensions{.pdf,.jpeg,.png}
\else
  % or other class option (dvipsone, dvipdf, if not using dvips). graphicx
  % will default to the driver specified in the system graphics.cfg if no
  % driver is specified.
  \usepackage[dvips]{graphicx}
  % declare the path(s) where your graphic files are
  % \graphicspath{{../eps/}}
  % and their extensions so you won't have to specify these with
  % every instance of \includegraphics
  \DeclareGraphicsExtensions{.eps}
\fi

\usepackage{amsmath}
% Not meant to use packages that *alter* fonts in IEEEtran
% but I think these just add symbols... so not sure if these are safe
\usepackage{amssymb,amsfonts,mathrsfs}
% Need to fix this back to the style's desired penalty
\interdisplaylinepenalty=2500

%\usepackage{algorithmic,algorithm}
% This one uses the improved algorithmicx with algorithmic-compatible commands
% see http://tex.stackexchange.com/questions/229355/algorithm-algorithmic-algorithmicx-algorithm2e-algpseudocode-confused
\usepackage{algpseudocode,algorithm}
\usepackage{tabularx,booktabs}

%\usepackage{caption}
%\captionsetup[table]{
%  labelsep = newline,
  %textfont = sc,
%  labelfont = small,
%  textfont = footnotesize,
%  name = TABLE, 
%  singlelinecheck=false,%%%%%%% a single line is centered by default
%  labelsep=colon,%%%%%%
%  justification=centerfirst,
%  skip = \medskipamount}

%\usepackage[
%font=small,
%labelsep=newline,
%justification=centerfirst,
%singlelinecheck=false % <-- important
%]{caption}

%\usepackage{array}
% Frank Mittelbach's and David Carlisle's array.sty patches and improves
% the standard LaTeX2e array and tabular environments to provide better
% appearance and additional user controls. As the default LaTeX2e table
% generation code is lacking to the point of almost being broken with
% respect to the quality of the end results, all users are strongly
% advised to use an enhanced (at the very least that provided by array.sty)
% set of table tools. array.sty is already installed on most systems. The
% latest version and documentation can be obtained at:
% http://www.ctan.org/pkg/array
%

% IEEEtran contains the IEEEeqnarray family of commands that can be used to
% generate multiline equations as well as matrices, tables, etc., of high
% quality.




% *** SUBFIGURE PACKAGES ***
%\ifCLASSOPTIONcompsoc
%  \usepackage[caption=false,font=footnotesize,labelfont=sf,textfont=sf]{subfig}
%\else
%  \usepackage[caption=false,font=footnotesize]{subfig}
%\fi
% subfig.sty, written by Steven Douglas Cochran, is the modern replacement
% for subfigure.sty, the latter of which is no longer maintained and is
% incompatible with some LaTeX packages including fixltx2e. However,
% subfig.sty requires and automatically loads Axel Sommerfeldt's caption.sty
% which will override IEEEtran.cls' handling of captions and this will result
% in non-IEEE style figure/table captions. To prevent this problem, be sure
% and invoke subfig.sty's "caption=false" package option (available since
% subfig.sty version 1.3, 2005/06/28) as this is will preserve IEEEtran.cls
% handling of captions.
% Note that the Computer Society format requires a sans serif font rather
% than the serif font used in traditional IEEE formatting and thus the need
% to invoke different subfig.sty package options depending on whether
% compsoc mode has been enabled.
%
% The latest version and documentation of subfig.sty can be obtained at:
% http://www.ctan.org/pkg/subfig

%
%\usepackage{stfloats}
% stfloats.sty was written by Sigitas Tolusis. This package gives LaTeX2e
% the ability to do double column floats at the bottom of the page as well
% as the top. (e.g., "\begin{figure*}[!b]" is not normally possible in
% LaTeX2e). It also provides a command:
%\fnbelowfloat
% to enable the placement of footnotes below bottom floats (the standard
% LaTeX2e kernel puts them above bottom floats). This is an invasive package
% which rewrites many portions of the LaTeX2e float routines. It may not work
% with other packages that modify the LaTeX2e float routines. The latest
% version and documentation can be obtained at:
% http://www.ctan.org/pkg/stfloats
% Do not use the stfloats baselinefloat ability as the IEEE does not allow
% \baselineskip to stretch. Authors submitting work to the IEEE should note
% that the IEEE rarely uses double column equations and that authors should try
% to avoid such use. Do not be tempted to use the cuted.sty or midfloat.sty
% packages (also by Sigitas Tolusis) as the IEEE does not format its papers in
% such ways.
% Do not attempt to use stfloats with fixltx2e as they are incompatible.
% Instead, use Morten Hogholm'a dblfloatfix which combines the features
% of both fixltx2e and stfloats:
%
% \usepackage{dblfloatfix}
% The latest version can be found at:
% http://www.ctan.org/pkg/dblfloatfix




%\ifCLASSOPTIONcaptionsoff
%  \usepackage[nomarkers]{endfloat}
% \let\MYoriglatexcaption\caption
% \renewcommand{\caption}[2][\relax]{\MYoriglatexcaption[#2]{#2}}
%\fi
% endfloat.sty was written by James Darrell McCauley, Jeff Goldberg and 
% Axel Sommerfeldt. This package may be useful when used in conjunction with 
% IEEEtran.cls'  captionsoff option. Some IEEE journals/societies require that
% submissions have lists of figures/tables at the end of the paper and that
% figures/tables without any captions are placed on a page by themselves at
% the end of the document. If needed, the draftcls IEEEtran class option or
% \CLASSINPUTbaselinestretch interface can be used to increase the line
% spacing as well. Be sure and use the nomarkers option of endfloat to
% prevent endfloat from "marking" where the figures would have been placed
% in the text. The two hack lines of code above are a slight modification of
% that suggested by in the endfloat docs (section 8.4.1) to ensure that
% the full captions always appear in the list of figures/tables - even if
% the user used the short optional argument of \caption[]{}.
% IEEE papers do not typically make use of \caption[]'s optional argument,
% so this should not be an issue. A similar trick can be used to disable
% captions of packages such as subfig.sty that lack options to turn off
% the subcaptions:
% For subfig.sty:
% \let\MYorigsubfloat\subfloat
% \renewcommand{\subfloat}[2][\relax]{\MYorigsubfloat[]{#2}}
% However, the above trick will not work if both optional arguments of
% the \subfloat command are used. Furthermore, there needs to be a
% description of each subfigure *somewhere* and endfloat does not add
% subfigure captions to its list of figures. Thus, the best approach is to
% avoid the use of subfigure captions (many IEEE journals avoid them anyway)
% and instead reference/explain all the subfigures within the main caption.
% The latest version of endfloat.sty and its documentation can obtained at:
% http://www.ctan.org/pkg/endfloat
%
% The IEEEtran \ifCLASSOPTIONcaptionsoff conditional can also be used
% later in the document, say, to conditionally put the References on a 
% page by themselves.

\usepackage{url}
\usepackage{dcolumn}
\newcolumntype{d}[1]{D{.}{.}{#1}}

\begin{document}

\title{Variable-Order de Bruijn Graphs}
% author names and IEEE memberships
% note positions of commas and nonbreaking spaces ( ~ ) LaTeX will not break
% a structure at a ~ so this keeps an author's name from being broken across
% two lines.
% use \thanks{} to gain access to the first footnote area
% a separate \thanks must be used for each paragraph as LaTeX2e's \thanks
% was not built to handle multiple paragraphs
%
%
%\IEEEcompsocitemizethanks is a special \thanks that produces the bulleted
% lists the Computer Society journals use for "first footnote" author
% affiliations. Use \IEEEcompsocthanksitem which works much like \item
% for each affiliation group. When not in compsoc mode,
% \IEEEcompsocitemizethanks becomes like \thanks and
% \IEEEcompsocthanksitem becomes a line break with idention. This
% facilitates dual compilation, although admittedly the differences in the
% desired content of \author between the different types of papers makes a
% one-size-fits-all approach a daunting prospect. For instance, compsoc 
% journal papers have the author affiliations above the "Manuscript
% received ..."  text while in non-compsoc journals this is reversed. Sigh.

\author{Alex~Bowe,
        Christina~Boucher,
        Travis~Gagie,
        Simon~J.~Puglisi,
        and~Kunihiko~Sadakane% <- this % stops a space
\IEEEcompsocitemizethanks{
\IEEEcompsocthanksitem A. Bowe is with the Department of Informatics, National Institute of Informatics, Japan. A. Bowe is a correponding author.
\ignore{\protect\\} E-mail: alex@nii.ac.jp
\IEEEcompsocthanksitem C. Boucher is with the Department of Computer Science, Colorado State University, Fort Collins, CO 80523-1873, USA. C. Boucher is a correponding author.
\ignore{\protect\\} E-mail: christina.boucher@colostate.edu
\IEEEcompsocthanksitem  T. Gagie, S.J.\ Puglisi are with the Department of Computer Science, University of Helsinki, Finland.
%\IEEEcompsocthanksitem  T. Gagie is with the School of Computer Science and Telecommunications, Diego Portales University, Chile.
%\IEEEcompsocthanksitem  S.J.\ Puglisi is with the Department of Computer Science, University of Helsinki, Finland.
\IEEEcompsocthanksitem K. Sadakane is with the Department of Mathematical Informatics, University of Tokyo, Japan.
}% <-this % stops a space
\thanks{}}
%\thanks{Manuscript received April 19, 2005; revised August 26, 2015.}}

%\IEEEcompsocitemizethanks{\IEEEcompsocthanksitem M. Shell was with the Department
%of Electrical and Computer Engineering, Georgia Institute of Technology, Atlanta,
%GA, 30332.\protect\\
% note need leading \protect in front of \\ to get a newline within \thanks as
% \\ is fragile and will error, could use \hfil\break instead.
%E-mail: see http://www.michaelshell.org/contact.html

% The paper headers
% The only time the second header will appear is for the odd numbered pages
% after the title page when using the twoside option.
\markboth{IEEE/ACM Transactions on Computational Biology and Bioinformatics}%
{Variable-Order de Bruijn Graphs}

% for Computer Society papers, we must declare the abstract and index terms
% PRIOR to the title within the \IEEEtitleabstractindextext IEEEtran
% command as these need to go into the title area created by \maketitle.
% As a general rule, do not put math, special symbols or citations
% in the abstract or keywords.
\IEEEtitleabstractindextext{%
\begin{abstract}
%\begin{abstract}
Modern genome sequencing is largely based on a process of randomly breaking replicated copies of a genome into fragments, using various technologies to capture the nucleotide sequence within these fragments (resulting in strings known as reads), and then using assembly software to attempt to reconstruct the original genome sequence from the reads.
This process is challenging as genomes contain repeated regions, and repeated regions much longer than read length confound assemblers, limiting their ability to completely and correctly reconstruct genomes successfully.
Correct and complete genome assembly is important because genomes encode elements that cooperate with others in close proximity, and thus not just the content, but  genome structure has important biological implications.
To the extent quality automated genome reconstruction is possible, there is an additional challenge of accessibility, as some of the most successful assembly software requires unusually high-end servers or clusters.
This limits their usefulness to biologists with access and skill to use such machines and hence more efficient computational techniques are of value.
Beyond efficiency and correctness of algorithms, there is interplay between computational approach, sequencing technology (which vary in read length, accuracy, applicability, and level of detail), and the assembly quality that may result.
In this report, we will expand on the concepts introduced here and review a selection of modern computational assembly tools, the sequence data on which they operate, and discuss important advantages, limitations, and possible extensions of them as well as their relationship to each other in the context of the sequence assembly problem.
%\end{abstract}

\end{abstract}

% Note that keywords are not normally used for peerreview papers.
%\begin{IEEEkeywords}
%Succinct data structures, Burrows-Wheeler transform, Genome assembly, de Bruijn graph
%\end{IEEEkeywords}

}

\maketitle


% To allow for easy dual compilation without having to reenter the
% abstract/keywords data, the \IEEEtitleabstractindextext text will
% not be used in maketitle, but will appear (i.e., to be "transported")
% here as \IEEEdisplaynontitleabstractindextext when the compsoc 
% or transmag modes are not selected <OR> if conference mode is selected 
% - because all conference papers position the abstract like regular
% papers do.
\IEEEdisplaynontitleabstractindextext
% \IEEEdisplaynontitleabstractindextext has no effect when using
% compsoc or transmag under a non-conference mode.



% For peer review papers, you can put extra information on the cover
% page as needed:
% \ifCLASSOPTIONpeerreview
% \begin{center} \bfseries EDICS Category: 3-BBND \end{center}
% \fi
%
% For peerreview papers, this IEEEtran command inserts a page break and
% creates the second title. It will be ignored for other modes.
\IEEEpeerreviewmaketitle

\chapter{Introduction}
\label{chp:introduction}

While consumer-grade genotyping - such as that used by 23andMe - has proven a popular and inexpensive method to determine Single Nucleotide Polymorphisms (SNPs) in individuals, such methods can only detect a set of reference genes, thus limiting their ability to detect all but the simplest variations.

Whole genome sequencing (without a reference) is a powerful alternative, albeit comparatively expensive. However, the price has been steadily declining: while the Human Genome Project cost \$2.7 billion to complete in 2003~\cite{HGP}, as of 2019 it is possible to have a genome sequenced for \$299~\cite{dantelabscost}, and the price continues to drop.

This decline in price is in large part owed to the advent of Next Generation Sequencing (NGS) machines. The “Sanger” sequencing method used in the Human Genome project required a high degree of human interaction, which NGS machines have subsequently automated, greatly increasing the speed and decreasing the cost. And although NGS machines produce much shorter reads (200 bases versus 800 bases in Sanger sequencing - a human genome is 3.4 billion bases), this is overcome by re-sequencing the same DNA.

%There has been another family of DNA sequencers appearing over the past eight years, which can read an entire chromosome at a time. However, they have unpredictable error profiles, making it difficult to sanitize the data, and it is unlikely this will improve without a major breakthrough in physics (cite). Consequently, instead of replacing NGS machines, they are often used in tandem by providing a reference when combining the short read data that NGS machines produce (cite).

The process of combining short reads into longer sequences is called assembly, and while finding the best overlap is NP-hard~\cite{Mye95}, many practical approaches have been proposed (see surveys \cite{KasMor06, MilKor10, Pop09}).

Traditionally, assembly employed an overlap graph, where each read is a node, and an edge exists if two reads have sufficient overlap~\cite{BatJaf02,HuaYan05,MyeSut00}. Assembly then involves computing a Hamiltonian tour of all nodes. This was an acceptable drawback when dealing with Sanger reads, but is prohibitively expensive when dealing with the abundant data that NGS machines produce.

Eulerian assembly~\cite{IW95, PTW} replaces the overlap graph with a de Bruijn graph, where every k-length substring of the reads is a node, and edges represent the k-1 length overlaps, where k is a user selected parameter. The contigs are then found by finding non-branching paths through this graph. Most modern assembler programs use this paradigm~\cite{bankevich2012spades,peng2010idba,Li:2010,Simpson:2009,Butler:2008,Zerbino:2008,SahShi12,MacPrz09}. See \cite{compeau11} for a thorough explanation of de Bruijn graphs and their use in assembly.

\includegraphics*[width=100ex]{images/graph-nodummies.pdf}
\includegraphics*[width=100ex]{images/cdbg.pdf}

While the de Bruijn graph can be constructed more efficiently than the overlap graph, it remains a bottleneck in assembly, both in terms of speed and size, with a de Bruijn graph of a human genome requiring 300 GB of RAM~\cite{Simpson:2009}. Previous work has reduced this to 30 GB~\cite{Conway}. This thesis reduces this to 2 GB, bringing it in line with commodity hardware - a student or field biologist could now perform this on their laptop. Around the same time as the work done in this thesis, an alternative approach with similar performance was published~\cite{wabi}, but the Burrows-Wheeler approach taken in this thesis offers more flexibility and faster edge traversal.

However, it is common for modern assemblers to build multiple de Bruijn graphs. This is because the k parameter significantly influences the topology - if k is too large, the vertices may not have edges, but if k is too small, the graph can become tangled (diagram). The perfect value of k is different for every set of reads, and in fact, due to non-uniform coverage of NGS data, different areas of the same graph may benefit from differing k values. Hence it has become common practice to build multiple graphs with increasing k, and use them in tandem (cite iterative dbg paper). The work in this thesis bypasses this iterative step, and introduces the first de Bruijn graph that can be built once, yet change k values on-the-fly, at only a modest increase in size over the base succinct de Bruijn graph (mention numbers).

Finally, in population genomics, biologists assemble multiple genomes in order to study the variations (give examples and cite). To avoid constructing multiple graphs, Iqbal et al. proposed the Colored de Bruijn Graph~\cite{ICTFM12}. This graph capitalizes on the fact that DNA is rarely unique to an individual. It does this by first constructing a de Bruijn Graph of the entire populations NGS reads, then, each individual is assigned a unique “color”, and the vertices and edges are annotated with the colors that they belong to. In this thesis, we further augment our succinct de Bruijn Graph to efficiently store these colors. When tested with four plant genomes, Iqbal’s structure required 101 GB RAM, while ours only requires 4 GB of RAM. Furthermore, our structure was able to store all known E. Coli genomes in 42 GB, where Iqbal’s was not able to complete, but is estimated to require 3 TB of RAM. We also demonstrate the use of our structure in creating a database of all Antimicrobial Resistance Genes, requiring 245 GB of RAM (an estimated 18 TB with Iqbal’s structure), for rapidly locating resilient bacterial outbreaks in food supply chains.

These three papers demonstrate that the burrows-wheeler approach is efficient, but can also be augmented to support extra queries that are commonplace in many modern assemblers. Due to the wealth of research on Burrows-Wheeler transforms and Suffix Arrays on which the data
structures in this thesis are based, it is likely that the set of supported operations will continue to grow as applications are found.

\section*{Original Work}
%Sections of this thesis are repeated verbatim from these original published papers:
%<each paper title, where it appeared, and a brief description>.
%Succinct de Bruijn Graphs:
%Variable Order de Bruijn Graphs:
%Succinct Colored de Bruijn Graphs:
%

\section{Preliminaries} \label{sec:preliminaries}

%DO WE WANT A FORMAL DEFINITION OF THE DEBRUIJN GRAPH IN HERE?
\subsection{De Bruijn Graphs} \label{sec:dbg}

Given an alphabet $\Sigma$ of $\sigma$ symbols and a set of strings $\lbrace
S_1, S_2, \ldots, S_t \rbrace$, $S_i \in \Sigma^{+}$, the {\em de Bruijn graph}
of order $K$, denoted $G^S_K$, or just $G_K$, when the context is clear, is a
directed, labelled graph defined as follows.

Let $M_{K}$ be the set of distinct $K$-mers (strings of length $K$) that occur
as substrings of some $S_i$. $M_{K+1}$ is defined similarly.  $G_K$ has exactly
$|M_{K}|$ nodes and with each node $u$ we associate a distinct $K$-mer from
$M_{K}$, denoted $\nodelabel(u)$. Edges are defined by $M_{K+1}$: for each
string $T \in M_{K+1}$ there is a directed edge, labelled with symbol $T[K+1]$,
from node $u$ to node $v$, where $\nodelabel(u) = T[1,K]$ and $\nodelabel(v) =
T[2,K+1]$. 

\subsection{Rank and Select} \label{sec:rank} Two basic operations used in almost
every succinct and compressed data structure are {\em rank} and {\em select}.
Given a sequence (string) $S[1,n]$ over an alphabet $\Sigma =
\{1,\ldots,\sigma\}$, a character $c \in \Sigma $, and integers $i$,$j$,
$\rank_c(S,i)$ is the number of times that $c$ appears in $S[1,i]$, and
$\select_c(S,j)$ is the position of the $j$-th occurrence of $c$ in $S$.
%There is a great variety of techniques to answer these queries, with
%suitability depending on the nature of the sequence, for example, on whether or
%not it will be compressed and on the size of the alphabet.
For a binary string $B[1,n]$, the classic solution for rank and
select~\cite{Mun96} is built upon the input sequence, requiring $o(n)$
additional bits.  Generally, $\rank_1$ and $\select_1$ are considered the
default rank and select queries.  More advanced solutions
(e.g.~\cite{bitvector}) achieve zero-order compression of $B$,
%For example, the several structures (e.g.~\cite{bitvector}), (see
%also~\cite{kkp2014}), 
representing it in just $nH_0(B) + o(n)$ bits of space, and supporting $\rank$
and $\select$ operations in constant time. 
%Several practical implementations and improvements of RRR exists (see,
%e.g.,~\cite{kkp2014}).

\subsection{Wavelet Trees} \label{sec:WVT} To support rank and select on larger
alphabet strings, the wavelet tree~\cite{ggv2003,n2013} is a commonly used data
structure that occupies $n\log\sigma + o(n\log\sigma)$ bits of space and
supports $\rank$ and $\select$ queries in $\Oh{\log\sigma}$ time.  Wavelet trees
also support a variety of more complex queries on the underlying string (see,
e.g.~\cite{gnp2012}), in $\Oh{\log\sigma}$ time, and we will make use of some of
this functionality in Section~\ref{sec:implementing}.
%One we will make use of in this paper is the {\em range successor query},
%$\rsucc_c(i,j)$, which returns the smallest character $d > c$ in $S[i,j]$, or
%$\inf$ if no such character exists.  Wavelet trees support \rsucc queries in
%$\Oh{\log\sigma}$ time.


%We will also make use of the Wavelet Tree~\cite{GGV03} data structure on a
%string $T=a_{1}a_{2} \ldots a_{n}$ over an alphabet $\Sigma$.
%
%The Wavelet Tree of $T$ is a binary balanced tree, where each leaf represents a
%symbol of $\Sigma$. The root is associated with the complete sequence $T$. Its
%left child is associated with a subsequence obtained by concatenating the
%symbols $a_i$ of $T$ satisfying $a_i < \sigma /2$. The right child corresponds
%to the concatenation of every symbol $a_i$ satisfying $a_i \geq \sigma$.  This
%relation is maintained recursively up to the leaves, which will be associated
%with the repetitions of a unique symbol.  At each node we store only a binary
%sequence of the same length of the corresponding sequence, using at each
%position a $0$ to indicate that the corresponding symbol is mapped to the left
%child, and a $1$ to indicate the symbol is mapped to the right child.
%
%If the bitmaps of the nodes support constant-time {\em rank} and {\em select}
%queries, then the Wavelet Tree support fast $access$, {\em rank} and {\em
%select} on $T$.
%
%\emph{Access:} In order to obtain the value of $a_{i}$ the algorithm begins at
%the root, and depending on the value of the root bitmap $B$ at position $i$, it
%moves down to the left or to the right child. If the bitmap value is $0$ it
%goes to the left, and replaces $i \leftarrow \rank_{0}(B,i)$. If the bitmap
%value is $1$ it goes to the right child and replaces $i \leftarrow
%\rank_{1}(B,i)$.  When a leaf is reached, the symbol associated with that leaf
%is the value of $a_i$.
%
%\emph{Rank:} To obtain the value of $\rank_c(S,i)$ the algorithm is similar: it
%begins at the root, and goes down updating $i$ as in the previous query, but
%the path is chosen according to the bits of $c$ instead of looking at $B[i]$.
%When a leaf is reached, the $i$ value is the answer.
%
%\emph{Select:} The value of $\select_c(S,j)$ is computed as follows: The
%algorithm begins in the leaf corresponding to the character $c$, and then moves
%upwards until reaching the root.  When it moves from a node to its parent, $j$
%is updated as $j \leftarrow \select_{0}(B,j)$ if the node is a left child, and
%$j \leftarrow \select_{1}(B,j)$ otherwise. When the root is reached, the final
%$j$ value is the answer.
%
%{\em Range Successor Queries}.



\section{BOSS representation}
\label{sec:BOSS}

% TODO: take the concise description of BOSS from CDBG paper?

Conceptually, to build the BOSS representation~\cite{bowe} of a $K$th-order de Bruijn graph from a set of \((K + 1)\)-mers, we first add enough dummy \((K + 1)\)-mers starting with \$s so that if \(\alpha a\) is in the set, then some \((K + 1)\)-mer ends with $\alpha$ ($\alpha$ a $K$-mer, $a$ a symbol).  We also add enough dummy \((K + 1)\)-mers ending with \$ that if \(b \alpha\) is in the set, with $\alpha$ containing no \$ symbols, then some \((K + 1)\)-mer starts with $\alpha$.  We then sort the set of \((K + 1)\)-mers into the right-to-left lexicographic order of their first $K$ symbols (with ties broken by the last symbol) to obtain a matrix.  If the $i$th through $j$th \((K + 1)\)-mers start with $\alpha$, then we say node \([i, j]\) in the graph has label $\alpha$, with \(j - i + 1\) outgoing edges labelled with the last symbols of the $i$th through $j$th \((K + 1)\)-mers.  If there are $n$ nodes in the graph, then there are at most \(\sigma n\) rows in the matrix, i.e., \((K + 1)\)-mers.

For example, if \(K = 3\) and the matrix is the one from Bowe et al.'s paper,
shown in the left of Fig.~\ref{fig:matrix}, then the \(n = 11\) nodes are
\begin{gather*}
 [1, 1], [2, 2], [3, 3], [4, 5], [6, 6], [7, 7], [8, 9], [10,
 10], [11, 11], \\ [12, 12], [13, 13]
\end{gather*}
with labels

\begin{gather*}
 \mathrm{\$\$\$}, \mathrm{CGA}, \mathrm{\$TA}, \mathrm{GAC}, \mathrm{TAC},
\mathrm{GTC}, \mathrm{ACG}, \mathrm{TCG}, \mathrm{\$\$T}, \\
\mathrm{ACT}, \mathrm{CGT},
\end{gather*}%
respectively. The 3rd-order de Bruijn graph itself is shown in the right of the figure.

\begin{figure*}[!t]
\centering
\begin{tabular}{c@{\hspace{10ex}}c}
\begin{tabular}{r@{\hspace{1ex}}@{\hspace{1ex}}@{\hspace{1ex}}l@{\hspace{1ex}}c}
1) & \,\$\,\$\,\$\, & T\\
2) & CGA & C\\
3) & \,\$\,TA & C\\
4) & GAC & G\\
5) & GAC & T\\
6) & TAC & G\\
7) & GTC & G\\
8) & ACG & A\\
9) & ACG & T\\
10) & TCG & A\\
11) & \,\$\,\$\,T & A\\
12) & ACT & \$\\
13) & CGT & C
\end{tabular} &
\raisebox{-10ex}
{\includegraphics*[trim = 0cm 0cm 9cm 24cm, width=50ex]{images/dbg-with-dummies.pdf}}
\end{tabular}
\caption{The BOSS matrix (left) and de Bruijn graph (right) for the quadruples CGAC, GACG, GACT, TACG, GTCG, ACGA, ACGT, TCGA, CGTC.}
\label{fig:matrix}
\hrulefill
\end{figure*}

Bowe et al.\ described a number of queries on the graph, all of which can be implemented in terms of the following three with at most an $\Oh{\sigma}$-factor slowdown:
\begin{itemize}
\item $\forward(v, a)$ returns the node $w$ reached from $v$ by an edge labelled $a$, or NULL if there is no such node;
\item $\backward(v)$ lists the nodes $u$ with an edge from $u$ to $v$;
\item $\lastchar(v)$ returns the last character of $v$'s label.
\end{itemize}
In our example, \(\forward \allowbreak ([8, 9], \mathrm{A}) \allowbreak =
\allowbreak [2, 2]\),
\(\backward \allowbreak ([2, 2]) \allowbreak = \allowbreak [8, 9], [10, 10]\) and
\(\lastchar \allowbreak ([8, 9]) \allowbreak = \allowbreak \mathrm{G}\).
Since $\backward$ always returns at least one node, we can recover any non-dummy node's entire label by $K$ calls to $\lastchar$ interleaved with \(K - 1\) calls to $\backward$.




\section{Varying order}
\label{sec:changing}

If we delete the first column of the matrix in Figure~\ref{fig:matrix}, the result is {\em almost} the BOSS matrix for a 2nd-order de Bruijn graph whose nodes
\[[1, 1], [2, 2], [3, 3], [4, 6], [7, 7], [8, 10], [11, 11], [12, 12], [13, 13]\]
have labels
\[\mathrm{\$\$, GA, TA, AC, TC, CG, \$T, CT, GT}\,,\]
respectively.  Similarly, if we delete the first two columns of the original matrix, the result is almost the BOSS matrix for a 1st-order graph whose nodes
\[[1, 1], [2, 3], [4, 7], [8, 10], [11, 13]\]
have labels
\[\mathrm{\$, A, C, G, T}\,,\]
respectively.  If we delete the first three columns, the result is almost the BOSS graph for the 0th-order graph whose single node \([1, 13]\) has an empty label.  Notice we allow the same node to appear in different graphs, with labels of different lengths.  If readers find this confusing, they can imagine that nodes are triples instead of pairs, with the additional component storing the label's length.

The truncated form of a higher order BOSS differs from the BOSS of a lower order in that
%The problem is that 
some rows are repeated, which could prevent the BOSS representation from working properly.  Suppose that, instead of trying to apply $\forward$, $\backward$ and $\lastchar$ directly to nodes in the new graphs, we augment the BOSS representation of the original graph to support the following three queries:
\begin{itemize}
\item $\shorter(v, k)$ returns the node whose label is the last $k$ characters of $v$'s label;
\item $\longer(v, k)$ lists nodes whose labels have length \(k \leq K\) and end with $v$'s label;
\item $\maxlen(v, a)$ returns some node in the original graph whose label ends with $v$'s label, and that has an outgoing edge labelled $a$, or NULL otherwise. %if there is no such node.
\end{itemize}
If we want a node in the original graph whose label ends with $v$'s label but we do not care about its outgoing edges, then we write \(\maxlen(v, *)\).  Notice $\shorter$ and $\longer$ are symmetric, in the sense that if $v$'s label has length $k_v$ and \(x \in \longer(v, k_v)\), then \(\shorter(x, k_v) = v\).  In our example, \(\shorter([4, 5], 2) = [4, 6]\) while \(\longer([4, 6], 3) = [4, 5], [6, 6]\) and \(\maxlen ([4, 6], \mathrm{G})\) could return either \([4, 5]\) or \([6, 6]\), while \(\maxlen([4, 6], \mathrm{T}) = [4, 5]\) and \(\maxlen([4, 6], \mathrm{A}) = \mathrm{NULL}\).

If $v$ is a node in the original graph --- e.g., $v$ is returned by $\maxlen$ --- then we can use the BOSS implementations of $\forward$, $\backward$ and $\lastchar$.  Otherwise, if $v$'s label has length $k_v$ then
\begin{eqnarray*}
\forward(v, a) & = & \shorter(\forward(\maxlen(v, a), a), k_v)\\
\lastchar(v) & = & \lastchar(\maxlen(v, *))\,.
\end{eqnarray*}
Assuming queries can be applied to lists of nodes, we can compute \(\backward(v)\) as %by computing
\[\shorter(\backward(\maxlen(\longer(v, k_v + 1), *)), k_v),\]
removing any duplicates.

To see why we can compute $\backward$ like this, suppose $v$'s label is \(\alpha a\), so \(\longer(v, \allowbreak k_v + 1)\) returns a list of all \(d \leq \sigma\) nodes whose labels have the form \(b \alpha a\).  Applying $\maxlen$ to this list returns a second list of $d$ nodes, with labels \(\beta_1 b_1 \alpha a, \ldots, \beta_d b_d \alpha a\) of length $K$.  Applying $\backward$ to this second list returns yet a third list, of all the at most \(\sigma d\) nodes whose labels have the form \(c \beta_i b_i \alpha\).  We need only one node returned calling $\backward$ on each node in the second list, so we can discard all but at most $d$ nodes in the third list.  Finally, applying $\shorter$ to the third list returns a fourth list, of all $d$ nodes whose labels have the form \(b_i \alpha\), each of which may be repeated at most $\sigma$ times in the list.



\section{Implementing $\shorter$, $\longer$ and $\maxlen$}
\label{sec:implementing}

The BOSS representation includes a wavelet tree over the last column $W$ of the BOSS matrix, and a bitvector $L$ of the same length with 1s marking where nodes' intervals end.  In our example, \(W = \mathrm{TCCGTGGATAA\$C}\) and \(L = 1110111011111\).

%With these data structures, 
Now we can implement \(\maxlen([i, j], a)\) in $\Oh{\log \sigma}$ time: we use $\rank$ and $\select$ on $W$ to find an occurrence \(W [r]\) of $a$ in \(W [i..j]\), if there is one; we then use $\rank$ and $\select$ on $L$ to find the last bit \(L [i' - 1] = 1\) with \(i' \leq r\) and the first bit \(L [j'] = 1\) with \(j' \geq r\), and return \([i', j']\).  (If there is no occurrence of 1 strictly before \(L [r]\), then we set \(i' = 1\).)  We can implement \(\maxlen([i, j], *)\) in $\Oh{1}$ time: instead of using $\rank$ and $\select$ on $W$ to find $r$, we simply choose any $r$ between $i$ and $j$.

In our example, for \(\maxlen([4, 6], \mathrm{G})\) we first find an occurence \(W [r]\) of G in \(W [4..6]\), which could be either \(W [4]\) or \(W [6]\); if we choose \(r = 4\) then the last bit \(L [i' - 1] = 1\) with \(i' \leq r\) is \(L [3]\) and the first bit \(L [j'] = 1\) with \(j' \geq r\) is \(L [5]\), so we return \([i', j'] = [4, 5]\); if we choose \(r = 6\) then the last bit \(L [i' - 1] = 1\) with \(i' \leq r\) is \(L [5]\) and the first bit \(L [j'] = 1\) with \(j' \geq r\) is \(L [6]\), so we return \([i', j'] = [6, 6]\).

To implement $\shorter$ and $\longer$, we store a wavelet tree over the sequence $L^*$ in which \(L^* [i]\) is the length of the longest common suffix of the label of the node in the original graph whose interval includes $i$, and the label of the node whose interval includes \(i + 1\); this takes $\Oh{\log K}$ bits per \((K + 1)\)-mer in the matrix.  To save space, we can omit $K$s in $L^*$, since they correspond to 0s in $L$ and indicate that $i$ and \(i + 1\) are in the interval of the same node in the original graph; the wavelet tree then takes $\Oh{\log K}$ bits per node in the original graph and $\Oh{n \log K}$ bits in total.  In our example, \(L^* = 0, 1, 0, 3, 2, 1, 0, 3, 2, 0, 1, 1\) (and we can omit the 3s to save space).

For \(\shorter([i, j], k)\), we use the wavelet tree over $L^*$ to find the largest \(i' \leq i\) and the smallest \(j' \geq j\) with \(L^* [i' - 1], L^* [j'] < k\) and return \([i', j']\), which takes $\Oh{\log K}$ time.  For \(\longer([i, j], k)\), we use the wavelet tree to find the set \(B = \{b\,:\,L^* [b] < k\,;\,i - 1 \leq b \leq j\}\) --- which includes \(i - 1\) and $j$ --- and then, for each consecutive pair \((b, b')\) in $B$, we report \([b + 1, b']\); this takes a total of $\Oh{|B| \log K}$ time.  With these implementations, if the time bounds for \(\forward(v, a)\), \(\backward(v)\) and \(\lastchar(v)\) are $\Oh{t_\forward}$, $\Oh{t_\backward}$ and $\Oh{t_\lastchar}$ when $v$ is a node in the original graph, respectively, then they are $\Oh{t_\forward + \log \sigma + \log K}$, $\Oh{\sigma (t_\backward + \log K)}$ and $\Oh{t_\lastchar + 1}$ when $v$ is not a node in the original graph.

In our example, for \(\shorter([4, 5], 2)\) we find the largest \(i' \leq 4\) and the smallest \(j' \geq 5\) with \(L^* [i' - 1], L^* [j'] < 2\) --- which are 4 and 6, respectively --- and return \([4, 6]\).  For \(\longer([4, 6], 3)\) we find the set \(B = \{b\,:\,L^* [b] < 3\,;\,3 \leq b \leq 6\} = \{3, 5, 6\}\) and report \([4, 5]\) and \([6, 6]\).

A smaller but slower approach is not to store $L^*$ explicitly but to support access to any cell \(L^* [i]\) by finding the nodes in the original graph whose intervals include $i$ and \(i + 1\), then using $\backward$ and $\lastchar$ to compute their labels and find the length of their longest common suffix; this takes a total of $\Oh{K (t_\backward + t_\lastchar)}$ time.  To implement $\shorter$ and $\longer$, we store a range-minimum data structure~\cite{fh2011} over $L^*$, which takes \(2 n + o (n)\) bits and returns the position of the minimum value in a specified substring of $L^*$ in $\Oh{1}$ time.

For \(\shorter([i, j], k)\), we use binary search and range-minimum queries to find the largest \(i' \leq i\) and the smallest
\(j' \geq j\) with \(L^* [i' - 1], L^* [j'] < k\) and return \([i', j']\), which takes $\Oh{K (t_\backward + t_\lastchar) \log (n \sigma)}$ time. 
%(With a more complicated use of the range-minimum data structure, which we will describe in the full version of this paper, we use $\Oh{K^2 (t_\backward + t_\lastchar)}$ time.)
For \(\longer([i, j], k)\), we recursively split \([i, j]\) into subintervals with range-minimum queries, at each step using $\backward$ and $\lastchar$ to check that the minimum value found is less than $k$; this takes $\Oh{K (t_\backward + t_\lastchar)}$ time per node returned.  With these implementations, \(\forward(v, a)\), \(\backward(v)\) and \(\lastchar(v)\) take $\Oh{t_\forward + K (t_\backward + t_\lastchar) \log (n \sigma)}$, $\Oh{\sigma K (t_\backward + t_\lastchar) \log (n \sigma) + \sigma^2 t_\backward}$ and $\Oh{t_\lastchar + 1}$ time, respectively, when $v$ is not a node in the original graph.

For \(\sigma = \Oh{1}\), our bounds are summarized in the following theorem. % We will provide more details in the full version of this paper.

\begin{theorem}
\label{thm:bounds}
When \(\sigma = \Oh{1}\), we can store a variable-order de Bruijn graph in $\Oh{n \log K}$ bits on top of the BOSS representation, where $n$
is the number of nodes in the $K$th-order de Bruijn graph, and support \forward\ and \backward\ in $\Oh{\log K}$ time and \lastchar\ in
$\Oh{1}$ time.  We can also use $\Oh{n}$ bits on top of the BOSS representation, at the cost of using $\Oh{K \log n / \log \log n}$ time
for \forward\ and \backward.
\end{theorem}

%shorter - logK
%longer - |B| log K time (for a set of B resulting nodes)

%K^2 log^2 n / log log n

%O(B K^2 log^2 n / log log n)

\section{Experiments}
\label{sec:experiments}

\begin{table}[t!]
%\caption{Construction Time and Memory Usage (top), and Mean Time per Navigation Operation (lower).}
\caption{Input size (top), construction time, memory use, and structure size (middle), and mean time taken for each 
navigation operation (lower), for all data sets and both structures. For variable-order, the multipliers in
parenthesis are the increase over the fixed-order results. Cells marked ``N/A'' for fixed-order indicate operations not
possible with that structure.
%The times in parentheses for $\longer$ are the mean times per node in the resulting set.
}
\scriptsize
\setlength\tabcolsep{1.8pt}
\begin{tabularx}{\textwidth}{@{}Rd{3.2}d{4.9}d{3.2}d{4.9}d{3.2}d{5.9}d{3.2}d{5.9}@{}}
% manual : http://ftp.jaist.ac.jp/pub/CTAN/macros/latex/required/tools/tabularx.pdf
						%\cline{2-9}
\toprule
{\bf Dataset}     & \multicolumn{2}{c}{{\em E.~coli}} & \multicolumn{2}{c}{Human chromosome 14} & \multicolumn{2}{c}{Human} 		& \multicolumn{2}{c}{Parrot} \\
% I think DSK size + number of K-mers is enough to demonstrate the increasing data set size
% I don't have times for DSK, so I'd have to run those again if needed
%Genome Size (bp) & \multicolumn{2}{c}{{4,639,221}} & \multicolumn{2}{c}{88,289,540} & \multicolumn{2}{c}{} 		& \multicolumn{2}{c}{} \\
%Number of Reads & \multicolumn{2}{c}{{}} & \multicolumn{2}{c}{} & \multicolumn{2}{c}{} 		& \multicolumn{2}{c}{} \\
%\midrule
%K & \multicolumn{2}{c}{{27}} & \multicolumn{2}{c}{55} & \multicolumn{2}{c}{55} 		& \multicolumn{2}{c}{55} \\
%Frequency Threshold & \multicolumn{2}{c}{{1}} & \multicolumn{2}{c}{1} & \multicolumn{2}{c}{2} 		& \multicolumn{2}{c}{1} \\
%DSK Time (mins) & \multicolumn{2}{c}{{}} & \multicolumn{2}{c}{} & \multicolumn{2}{c}{} 		& \multicolumn{2}{c}{} \\
{\bf DSK Size (GB)} & \multicolumn{2}{c}{{1.52}} & \multicolumn{2}{c}{6.88} & \multicolumn{2}{c}{26.74} 		& \multicolumn{2}{c}{70.28} \\
{\bf\boldmath Number of $K$-mers} & \multicolumn{2}{c}{{204,098,902}} & \multicolumn{2}{c}{461,445,333} & \multicolumn{2}{c}{1,794,522,954} 		& \multicolumn{2}{c}{4,716,731,435} \\
{\bf BOSS Order} & \multicolumn{1}{c}{fixed} 	& \multicolumn{1}{c}{variable} & \multicolumn{1}{c}{fixed} 	& \multicolumn{1}{c}{variable} & \multicolumn{1}{c}{fixed} 	& \multicolumn{1}{c}{variable} & \multicolumn{1}{c}{fixed} 	& \multicolumn{1}{c}{variable} \\
\midrule
%\cline{1-9}
{\bf Construction (mins)} & 3.93 & 5.09 \enspace (1.30{\sf x}) & 14.37 & 18.72 \enspace (1.30{\sf x}) & 64.45 & 83.85 \enspace (1.30{\sf x})& 162.58 & 225.73 \enspace (1.39{\sf x})\\
{\bf Graph Size (GB)}  			   & 0.16  & 0.41 \enspace (2.56{\sf x})  & 0.40   & 1.38 \enspace (3.45{\sf x}) & 1.67 & 5.42 \enspace (3.25{\sf x}) & 4.20 & 13.60 \enspace (3.24{\sf x}) \\
{\bf Peak RAM (GB)}  		 & 3.16 & 3.16 \enspace (1.00{\sf x}) & 3.22 & 3.22 \enspace (1.00{\sf x})& 7.65 & 9.31 \enspace (1.22{\sf x}) & 15.30 & 15.29 \enspace (1.00{\sf x}) \\
{\bf Peak Disk (GB)}  	 & 12.17 & 12.17 \enspace (1.00{\sf x}) & 56.68 & 56.68 \enspace (1.00{\sf x}) & 248.37 & 248.37 \enspace (1.00{\sf x}) & 562.28 & 562.28 \enspace (1.00{\sf x})\\
%\cline{1-9}
\midrule
$\forward$ ($\mu$s)   & 6.00 & 17.03 \enspace (2.84{\sf x}) &6.24	&16.17 \enspace (2.59{\sf x}) &7.07	&18.31 \enspace (2.59{\sf x})&7.77	 &19.39 \enspace (2.50{\sf x})\\
$\backward$ ($\mu$s)  & 8.23 & 59.77 \enspace (7.26{\sf x}) &8.47	&55.63 \enspace (6.57{\sf x}) &9.27	&62.85 \enspace (6.78{\sf x})&10.46 &63.87 \enspace (6.11{\sf x})\\
$\lastchar$ ($\mu$s)  & 0.01 &  0.01 \enspace (1.00{\sf x}) &0.01	& 0.01 \enspace (1.00{\sf x}) &0.01	& 0.01 \enspace (1.00{\sf x})&0.01	 &0.01 \enspace (1.00{\sf x})\\

$\maxlen$ ($\mu$s)    &\multicolumn{1}{r}{N/A}  & 1.43   &\multicolumn{1}{r}{N/A} &1.56	   &\multicolumn{1}{r}{N/A}  	&2.02	    &\multicolumn{1}{r}{N/A}  &2.46 \\ 
$\maxlen_c$ ($\mu$s)  &\multicolumn{1}{r}{N/A}  &	5.41  &\multicolumn{1}{r}{N/A} &5.98	   &\multicolumn{1}{r}{N/A}  &6.71	    &\multicolumn{1}{r}{N/A}  &7.49 \\
$\shorter_1$ ($\mu$s) &\multicolumn{1}{r}{N/A}   &	14.65 &\multicolumn{1}{r}{N/A} &17.72	 &\multicolumn{1}{r}{N/A} 	  &19.54	  &\multicolumn{1}{r}{N/A}  &19.84 \\
$\shorter_2$ ($\mu$s) &\multicolumn{1}{r}{N/A}  &	14.83 &\multicolumn{1}{r}{N/A} &17.79	 &\multicolumn{1}{r}{N/A}  	&19.68	  &\multicolumn{1}{r}{N/A}  &19.98 \\
$\shorter_4$ ($\mu$s) &\multicolumn{1}{r}{N/A}   &	15.11 &\multicolumn{1}{r}{N/A} &18.02	 &\multicolumn{1}{r}{N/A}   &19.90	  &\multicolumn{1}{r}{N/A}  &20.20 \\
$\shorter_8$ ($\mu$s) &\multicolumn{1}{r}{N/A}   &	15.73 &\multicolumn{1}{r}{N/A} &18.39	 &\multicolumn{1}{r}{N/A}   &20.29	  &\multicolumn{1}{r}{N/A}  &20.64 \\
%$\shorter$ ($\mu$s)   &\multicolumn{1}{r}{N/A}   &	15.08 &\multicolumn{1}{r}{N/A} &17.98	 &\multicolumn{1}{r}{N/A}   &19.85	  &\multicolumn{1}{r}{N/A}  &20.17 \\ % mean avg of above 4
$\longer_1$ ($\mu$s)  &\multicolumn{1}{r}{N/A}   &21.53   &\multicolumn{1}{r}{N/A} &18.61	 &\multicolumn{1}{r}{N/A}   &21.06	  &\multicolumn{1}{r}{N/A}  &20.57 \\ 
$\longer_2$ ($\mu$s)  &\multicolumn{1}{r}{N/A}  &56.96   &\multicolumn{1}{r}{N/A} &41.08	 &\multicolumn{1}{r}{N/A}   &49.01	  &\multicolumn{1}{r}{N/A}  &47.07\\
$\longer_4$ ($\mu$s)  &\multicolumn{1}{r}{N/A}   &503.60  &\multicolumn{1}{r}{N/A} &323.50	 &\multicolumn{1}{r}{N/A}   &446.51	  &\multicolumn{1}{r}{N/A}  &428.97 \\
$\longer_8$ ($\mu$s)  &\multicolumn{1}{r}{N/A}  &6441.33 &\multicolumn{1}{r}{N/A}  &5338.38 &\multicolumn{1}{r}{N/A}   &18349.80	&\multicolumn{1}{r}{N/A}  &24844.80 \\

%$\longer_1$ ($\mu$s)  &\multicolumn{1}{r}{N/A} &21.53 \enspace (10.82)  &\multicolumn{1}{r}{N/A} &18.61 \enspace (12.38)  &\multicolumn{1}{r}{N/A} &21.06 \enspace (13.25)    &\multicolumn{1}{r}{N/A} &20.57 \enspace (12.99)\\ 
%$\longer_2$ ($\mu$s)  &\multicolumn{1}{r}{N/A} &56.96 \enspace (8.74)   &\multicolumn{1}{r}{N/A} &41.08 \enspace (11.24)  &\multicolumn{1}{r}{N/A} &49.01 \enspace (11.74)    &\multicolumn{1}{r}{N/A} &47.07 \enspace (11.02)\\
%$\longer_4$ ($\mu$s)  &\multicolumn{1}{r}{N/A} &503.60 \enspace (7.98)  &\multicolumn{1}{r}{N/A} &323.50 \enspace (10.59) &\multicolumn{1}{r}{N/A} &446.51 \enspace (11.04)   &\multicolumn{1}{r}{N/A} &428.97 \enspace (9.95)\\
%$\longer_8$ ($\mu$s)  &\multicolumn{1}{r}{N/A} &6441.33 \enspace (6.44) &\multicolumn{1}{r}{N/A} &5338.38 \enspace (9.03) &\multicolumn{1}{r}{N/A} &18349.80 \enspace (10.09) &\multicolumn{1}{r}{N/A} &24844.80 \enspace (9.48)\\

%$\longer_1$ nodes&	N/A	39.80	N/A	30.06	N/A	31.78	N/A	31.66
%$\longer_2$ nodes&	N/A	130.32	N/A	73.11	N/A	83.48	N/A	85.42
%$\longer_4$ nodes&	N/A	1262.84	N/A	610.77	N/A	809.06	N/A	861.91
%$\longer_8$ nodes&	N/A	20010.15	N/A	11824.04	N/A	36370.68	N/A	52388.38
\bottomrule
%\cline{2-9}
\end{tabularx}
% $\dagger$ $\lastchar$ is a reverse lookup in a very small array, so the speed is in fractions of a nanosecond.
% $\ddagger$ $\longer$ is reported as the average time  for the inputs $1$,$2$,$4$,$8$, due to much longer calculation time. }
\label{tab:nav-time}
\end{table}

%We have implemented the faster version of our data structure on top of an efficient implementation of the BOSS single-$K$ data structure\footnote{The
We have implemented the wavelet tree based data structure on top of an efficient implementation of the BOSS single-$K$ data structure\footnote{The
implementation is released under GPLv3 license at \url{http://github.com/cosmo-team/cosmo}. As Cosmo is under continuous development,
a static snapshot of the code used in this paper is available at \url{https://github.com/cosmo-team/cosmo/tree/varord-paper}.}.
Both structures make use of the SDSL-lite software library\footnote{\url{https://github.com/simongog/sdsl-lite}} for succinct data structures, and the
the construction code makes use of the STXXL software library\footnote{\url{https://github.com/stxxl/stxxl}} for external memory data structures and sorting.
The construction code is also concurrent in many places.%, making use of C++11 threads, and OpenMP\footnote{\url{http://openmp.org/}} for parallel internal memory sorting.
The smaller but slower version was not implemented.
%Further implementation details are described in \ref{sec:implementation}.

% TODO : cite their algorithmic paper?
% What makes STXXL good:
% http://stxxl.sourceforge.net/tags/master/introduction.html
% http://stxxl.sourceforge.net/tags/master/design.html
% http://stxxl.sourceforge.net/tags/master/design_algo_sorting.html
% http://stxxl.sourceforge.net/tags/master/citelist.html

% System
Our test machine was a server with a hyperthreaded quad-core 2.93 Ghz Intel Core i7-875K CPU and 16 GB RAM running
Ubuntu Server 14.04. Four Samsung 850 EVO 250GB SSDs were used for temporary storage for STXXL,
with a fifth identical drive used for temporary storage for SDSL-Lite and final graph output. In order to make
use of STXXL's parallel disk and asynchronous I/O support\footnote{\url{http://stxxl.sourceforge.net/tags/master/design_algo_sorting.html}},
the SSDs were not in a RAID configuration. The input files were read from a mechanical 2TB 7200 RPM disk.
% I might have been able to load the DSK files from a SSD thinking back... but it wouldn't make a difference except faster run time *for everything*

To minimize the effect of external factors on our results, each experiment was repeated three times with the minimum values
reported. The swap file was disabled, forcing the operating system to keep each graph completely in memory, and
there were no other users on the server.

% TODO : should construction be moved up and merged with the above?
% TODO : describe the construction algorithm
% TODO : describe why we needed external construction

\subsection{Test Data}

%To assess assembly quality, we aligned the reads to the {\em E.~coli} reference genome (substr.  K-12) using BWA (version 0.5.9) \cite{li2009fast} with default parameters.  We call a read {\em mapped} if BWA outputs an alignment for it and {\em unmapped} otherwise.  Analysis of the alignments revealed that 98\% of the reads mapped to the reference genome, representing an average depth of approximately $600\times$.  Next, we determined the amount of memory and time needed for our method for a larger dataset.  For that, we 
%The reference genome was also downloaded from the website (Reference genome GCA\_000001405.16).  
% Analysis of the alignments revealed that 98\% of the reads mapped to the reference genome, which represents 
% approximately 47x coverage of the genome.

In order to test the scalability of our approach, we repeated the experiment on readsets of varying size.
Our first data set consists of 27 million paired-end 100 character reads (strings)
from {\em E.~coli} (substr.  K-12). It was obtained from the NCBI Short Read Archive (accession 
ERA000206, EMBL-EBI Sequence Read Archive). The total size of this data set is around 2.3 GB compressed on disk (6 GB uncompressed).

The second data set is 36 million 155 character reads from the Human chromosome 14 Illumina reads used in the GAGE
benchmark\footnote{\url{http://gage.cbcb.umd.edu/}}, totalling 1.3 GB compressed on disk (6 GB uncompressed).

For our third data set we obtained 1,415 million  paired-end 100 character Human genome reads
(SRX01231) that were generated by Illumina Genome Analyzer (GA) IIx platform. The total size of this data set is 130 GB compressed on disk (470 GB uncompressed).

Our fourth data set is 700 million paired-end 101 character reads, and 131 paired-end 75 character reads from the short
insert libraries of the Parrot data (ERA201590) provided in Assemblathon 2\cite{assemblathon2}.
The total size of this data set is 64 GB compressed on disk (245 GB uncompressed).

%We ran DSK on each... The human data set required too much external memory, and the final dBG size would have been too large to fit in internal memory,
%so had to be re-run with the frequency threshold set to 2.

We used DSK~\cite{dsk} on each data set to find the unique $(K+1)$-mers.
It is usual to have DSK ignore low-frequency $(K+1)$-mers (as they may result from sequencing errors). 
However, removing such $(K+1)$-mers may result in the removal of some $k$-mers with $k \leq K$ that would 
otherwise have an acceptable frequency. We therefore set the frequency threshold to be as low as possible: $1$ (accepting 
all $(K+1)$-mers) for all data sets except for the Human genome data set, which was too big for our SSDs during construction,
and too big to fit into RAM afterwards. Hence, for the Human genome data set, the frequency threshold was $2$.

%\footnote{This increases the resulting size, but the size relative to the standard BOSS representation can still be compared.
%Variable-$K$ frequency filtering can be implemented using the $L^*$ vector, but will affect the dummy edge calculations, and
%is hence designated as future work.}.
% TODO: mention mercy kmers (Megahit) as another strategy? ultimately we just don't want gaps

A value of $K = 27$ was chosen
%\footnote{Note that the BOSS de Bruijn graph is defined in terms of $K+1$-mers, so DSK must be run for $K+1$ (i.e. 28 and 56) rather than for $K$.}
for the {\em E.~coli} data, and $K = 55$
for the Human data sets as these values produced good assemblies in previous papers (see, e.g.,~\cite{paul}). $K = 55$ was also chosen for the Parrot
data set, %as there was no clear choice for $K$ from the Assemblathon 2 contestants, and
as it produced a graph that almost filled the main memory.
The resulting file sizes and $(K+1)$-mer totals are shown in Table~\ref{tab:nav-time}.

% TODO: Run experiment with one data set over multiple Ks to see how it scales that way

% TODO: Time DSK (multithreaded) (I don't have timings for these)
%DSK took 28 and 58 minutes to run on the {\em E.~coli} and human data sets, respectively.

\subsection{Construction}
\label{sec:construction}

In order to convert the input DSK data to the format required by BOSS (in the correct order, with dummy edges, as required by both single-$K$ and variable-$K$ structures),
we use the following process, which has been designed with disk I/O in mind.

While reading the DSK input data, we generate and add the reverse complements for each $(K+1)$-mer, then sort them by their first $K$ symbols (the source nodes). Concurrently, we also sort another copy of the $(K+1)$-mers and their reverse complements by their last $K$ symbols (the target nodes). Let the resulting tables
be $A$ and $B$, respectively.

Next, we calculate the set differences $A-B$, comparing only the $K$-length prefixes to the $K$-length suffixes respectively. This tells us which source nodes do not
appear as target nodes, which we prepend with $\$$ signs to create the required incoming dummy edges ($K$ each), and then sort by the first $K$ symbols. Concurrently, we also calculate $B-A$ to give us
the nodes requiring outgoing dummy edges (to which we append $\$$). Let the resulting tables be $I$ and $O$, respectively. At this point $B$ can be deleted.
%\footnote{This will create duplicate strings, which can be avoided using Longest Common Prefix calculations.}.

Finally, we perform a three-way merge (by first $K$ symbols) of $A$, $I$, and $O$, outputting the rightmost column. In the case of the variable-$K$ graph,
we also calculate the $L^{*}$ values while merging. Finally, we construct the necessary succinct indexes from the output.

The time bottleneck in the above process is clearly in sorting the $A$ and $I$ tables. $|I|$ can be as big as $K|A|$, but in practice only $1\%$ or fewer
nodes require incoming dummies. Our elements are of size $\Oh{K}$, thus, overall, construction of both data structures takes $\Oh{K^2|A|\log|A|}$ time
and $\Oh{K^2|A|}$ space in theory, but in practice takes $\Oh{K|A|\log|A|}$ time and $\Oh{K|A|}$ space.

%The large file sizes necessitated an external construction scheme.

% Draw attention to the limited difference
%The only place that construction differs for the variable order de Bruijn graph is during the merge, where the longest common suffix length is calculated
%for consecutive edges, and written to disk. Finally, when constructing the rank and select structures, the variable order de Bruijn graph creates a Wavelet Tree
%as well.


\subsection{Results}

For each data set, the $(K+1)$-mers from DSK (and their reverse complements) were converted into the BOSS format
using the process outlined in \ref{sec:construction}, using the external memory vectors and multithreaded, external
memory sort from STXXL. The BOSS structure and $L^{*}$ wavelet tree were then built using indexes from SDSL-lite.

Construction times and structure sizes are shown in Table~\ref{tab:nav-time}.
While the variable-$K$ BOSS structure is around $30\%$ slower to build, and $2.6$ to $3.5$ times larger 
than the standard BOSS structure, this is clearly much faster and less space consuming than building 
$K$ separate instances of the BOSS structure. The peak RAM and disk usage is the same for both structures
except in the case of the Human genome data set, where the variable-$K$ BOSS structure used $22\%$ more RAM.
% TODO : repeat, if same results, trace RAM usage, work out why?

% identical peak disk and RAM
% TODO : Does this sounds week since we didn't measure building/storing each static-k dbg individually?
% TODO : Should we compare it to megahit?

%% TODO: fix the xs in this table? \times doesnt handle \em well though
%\begin{table}[h!]
%\begin{tabularx}{\textwidth}{@{\extracolsep{\fill} } r  c  c   c  c }
%						& \multicolumn{2}{c}{{\em Escherichia coli}} 		& \multicolumn{2}{c}{Human chromosome 14} \\
%						\cline{2-5}
%   						& BOSS 		& multi-K BOSS			& BOSS  		&  multi-K BOSS  \\
%\hline
%Wall Time (mins) & 19  & 25 {\em(1.32x)} & 153 & 203 {\em(1.33x)} \\
%Final Size (MB)  & 163 & 420 {\em(2.58x)} & 414 & 1416 {\em(3.42x)}\\
%Genome Size (bp) 	&  \multicolumn{2}{c}{4,639,221} 			&  \multicolumn{2}{c}{88,289,540} \\
%Number of Reads 			&  \multicolumn{2}{c}{27 M} 				&  \multicolumn{2}{c}{36.5 M}  \\
%DSK Time (mins) 	&  \multicolumn{2}{c}{28} 			&  \multicolumn{2}{c}{58} \\
%\hline
%\end{tabularx}
%\caption{Summary of data sets, as well as construction time and final space for BOSS de Bruijn graph and multi-$K$ de Bruijn graph. For the multi-$K$ BOSS representation, the
%increase factor is shown in parentheses.}
%\label{tab:build}
%\end{table}

%Due to the non-trivial engineering effort required to integrate a particular de Bruijn graph 
%structure in a live assembler, we compare our new data structure to the original BOSS structure
%by measuring average times for navigation operations.
To measure navigation functions $\forward$ and $\backward$ we took the mean time 
over 20,000 random queries. For the variable-$K$ graph, the $k$ values for each node 
were chosen randomly between $8$ and $K$. 
%(involves only a reverse lookup on a very small array). 
Results are shown in Table~\ref{tab:nav-time}. 
The new structure makes the $\forward$ operation $2.5$ to $3$ times slower for $k < K$, though we 
note that for $k = K$ $\forward$ time is identical.
The $\backward$ operation is much slower in the new structure, but is much less frequently 
used than $\forward$ in assembly algorithms (for a variation that supports fast
$\backward$ calculations, see \cite{varorder-latin}). We also measured $\lastchar$, which took only
nanoseconds on both structures.

To see how fast the order can be changed, we timed $\shorter$ and $\longer$ for
changes of $1$, $2$, $4$, and $8$ symbols. Our experiments show that in practice changing 
order by a single symbol ($\shorter_1$ and $\longer_1$) is a cheap operation, taking around the
same time as $\forward$. For larger changes in order, the time for $\shorter$ is stable
($\shorter_1$, $\shorter_2$, $\shorter_4$, and $\shorter_8$ all take roughly the same time),
whereas $\longer$ takes significantly more time as the difference in order increases. This is because
$\longer$ must compute a set of nodes, and the size of that set grows roughly exponentially with
the change in order ($\longer$ takes around $10 \mu$s per node when averaged over the size of the resulting set).
%If not every node in the result is to be visited, the set could instead be calculated lazily.

As expected, $\maxlen$ is very fast (it requires a single rank and select operation
over a bit vector), and only slightly affected when finding the specified outgoing edge label (which
uses a rank and select over the BOSS wavelet tree instead).

% TODO: add separate measurements for 2,4,8... but as order changes increase, the... factor is more evident.



\chapter{Conclusions}
\label{sec:conclusions}

%\section{Future Work}
%\label{sec:future-work}


% An example of a floating figure using the graphicx package.
% Note that \label must occur AFTER (or within) \caption.
% For figures, \caption should occur after the \includegraphics.
% Note that IEEEtran v1.7 and later has special internal code that
% is designed to preserve the operation of \label within \caption
% even when the captionsoff option is in effect. However, because
% of issues like this, it may be the safest practice to put all your
% \label just after \caption rather than within \caption{}.
%
% Reminder: the "draftcls" or "draftclsnofoot", not "draft", class
% option should be used if it is desired that the figures are to be
% displayed while in draft mode.
%
%\begin{figure}[!t]
%\centering
%\includegraphics[width=2.5in]{myfigure}
% where an .eps filename suffix will be assumed under latex, 
% and a .pdf suffix will be assumed for pdflatex; or what has been declared
% via \DeclareGraphicsExtensions.
%\caption{Simulation results for the network.}
%\label{fig_sim}
%\end{figure}

% Note that the IEEE typically puts floats only at the top, even when this
% results in a large percentage of a column being occupied by floats.
% However, the Computer Society has been known to put floats at the bottom.


% An example of a double column floating figure using two subfigures.
% (The subfig.sty package must be loaded for this to work.)
% The subfigure \label commands are set within each subfloat command,
% and the \label for the overall figure must come after \caption.
% \hfil is used as a separator to get equal spacing.
% Watch out that the combined width of all the subfigures on a 
% line do not exceed the text width or a line break will occur.
%
%\begin{figure*}[!t]
%\centering
%\subfloat[Case I]{\includegraphics[width=2.5in]{box}%
%\label{fig_first_case}}
%\hfil
%\subfloat[Case II]{\includegraphics[width=2.5in]{box}%
%\label{fig_second_case}}
%\caption{Simulation results for the network.}
%\label{fig_sim}
%\end{figure*}
%
% Note that often IEEE papers with subfigures do not employ subfigure
% captions (using the optional argument to \subfloat[]), but instead will
% reference/describe all of them (a), (b), etc., within the main caption.
% Be aware that for subfig.sty to generate the (a), (b), etc., subfigure
% labels, the optional argument to \subfloat must be present. If a
% subcaption is not desired, just leave its contents blank,
% e.g., \subfloat[].


% An example of a floating table. Note that, for IEEE style tables, the
% \caption command should come BEFORE the table and, given that table
% captions serve much like titles, are usually capitalized except for words
% such as a, an, and, as, at, but, by, for, in, nor, of, on, or, the, to
% and up, which are usually not capitalized unless they are the first or
% last word of the caption. Table text will default to \footnotesize as
% the IEEE normally uses this smaller font for tables.
% The \label must come after \caption as always.
%
%\begin{table}[!t]
%% increase table row spacing, adjust to taste
%\renewcommand{\arraystretch}{1.3}
% if using array.sty, it might be a good idea to tweak the value of
% \extrarowheight as needed to properly center the text within the cells
%\caption{An Example of a Table}
%\label{table_example}
%\centering
%% Some packages, such as MDW tools, offer better commands for making tables
%% than the plain LaTeX2e tabular which is used here.
%\begin{tabular}{|c||c|}
%\hline
%One & Two\\
%\hline
%Three & Four\\
%\hline
%\end{tabular}
%\end{table}


% Note that the IEEE does not put floats in the very first column
% - or typically anywhere on the first page for that matter. Also,
% in-text middle ("here") positioning is typically not used, but it
% is allowed and encouraged for Computer Society conferences (but
% not Computer Society journals). Most IEEE journals/conferences use
% top floats exclusively. 
% Note that, LaTeX2e, unlike IEEE journals/conferences, places
% footnotes above bottom floats. This can be corrected via the
% \fnbelowfloat command of the stfloats package.

\section*{Acknowledgments}
We thank two anonymous reviewers for thoughtful comments that materially improved this manuscript.


% Can use something like this to put references on a page
% by themselves when using endfloat and the captionsoff option.
\ifCLASSOPTIONcaptionsoff
  \newpage
\fi



% trigger a \newpage just before the given reference
% number - used to balance the columns on the last page
% adjust value as needed - may need to be readjusted if
% the document is modified later
%\IEEEtriggeratref{8}
% The "triggered" command can be changed if desired:
%\IEEEtriggercmd{\enlargethispage{-5in}}

\bibliographystyle{IEEEtran}
\bibliography{IEEEabrv,dbg}

% You can push biographies down or up by placing
% a \vfill before or after them. The appropriate
% use of \vfill depends on what kind of text is
% on the last page and whether or not the columns
% are being equalized.

%\vfill

% Can be used to pull up biographies so that the bottom of the last one
% is flush with the other column.
%\enlargethispage{-5in}

\end{document}


