\documentclass{letter}

\makeatletter
\renewcommand{\closing}[1]{\par\nobreak\vspace{\parskip}%
  \stopbreaks
  \noindent
  \ifx\@empty\fromaddress\else
  \hspace*{\longindentation}\fi
  \parbox{\indentedwidth}{\raggedright
       \ignorespaces #1\\[3\medskipamount]%
       \ifx\@empty\fromsig
           \fromname
       \else \fromsig \fi\strut}%
   \par}
\makeatother

\signature{Alex Bowe}
%PhD candidate, Department of Informatics \\
%National Institute of Informatics \\
%2-1-2 Hitotsubashi, Chiyoda-ku, Tokyo, 101-8430 \\
%+81-3-4212-2000 \\
%alex@nii.ac.jp \\
\begin{document}
\begin{letter}{Dr. Ying Xu \\ Editor-in-Chief \\ IEEE Transactions on Computational Biology and Bioinformatics}
\opening{Dear Dr. Xu,}

We are submitting this manuscript entitled \textit{Variable-Order de Bruijn Graphs}
for publication as a short paper in IEEE Transactions on Computational Biology
and Bioinformatics.

In this manuscript, we show how to augment a succinct de Bruijn graph
representation by Bowe et al. (Proc. WABI, 2012) to support new operations that
let us change order on the fly, effectively representing all de Bruijn graphs up
to some maximum order $K$ in a single data structure. Our experiments show
our variable-order de Bruijn graph only modestly increases space usage,
construction time, and navigation time compared to a single order graph.

This research is relevant to the scope of IEEE Transactions on Computational Biology
and Bioinformatics, as it removes the iterative construction bottleneck of
state-of-the-art DNA assemblers, which must construct many single order
de Bruijn graphs.

This manuscript extends our previous experimental analysis (Proc. DCC, IEEE, 2015)
by using a new external construction algorithm to demonstrate the scalability of
our approach on much larger data sets. It is not under consideration by any other
journal.

All authors have approved this submission, and there are no conflicts of
interest. We would like to suggest Rayan Chikhi, Gregory Kucherov, Eric Rivals,
and Gustavo Sacomoto as suitably qualified reviewers.

Thank you for your time and effort in considering our work.

\closing{Sincerely,}
\end{letter}
\end{document}

