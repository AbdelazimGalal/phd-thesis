The de Bruijn graph $G_K$ of a set of strings $S$ is a key data structure in 
genome assembly that represents overlaps between all the $K$-length substrings 
of $S$. Construction and navigation of the graph is a space and time bottleneck 
in practice and the main hurdle for assembling large genomes. 
This problem is compounded because state-of-the-art assemblers do not build the de Bruijn graph for a single order (value of $K$) but for multiple values of $K$: they build $d$ de Bruijn graphs, each with a specific order, i.e., $G_{K_1}, G_{K_2}, \ldots, G_{K_d}$.  
Although, this paradigm increases the quality of the assembly produce but it greatly increases runtime, because of the need to construct $d$ graphs instead of one.
%A basic question when using de  Bruijn graph is how to choose the order of the graph, $K$, such that most nodes  have small but positive numbers of outgoing edges.
%In an ideal scenario, multiple orders (values of $K$) would be used for the graph construction.  In regions where there exists larger repeats  However, the optimal order (value of $K$) for the de Bruijn graph can be difficult to determine since it may vary across the genome.
%Because of genomic repetition  and other vagaries of the DNA sequencing process, however, the optimal order  can be difficult to determine, and so current assemblers build several  graphs, each of different order, which drastically increases assembly time. 
%erent for different parts of the DNA sequence. We could store several graphs of different orders and switch back and forth between them, but this might take too much space.  
In this paper, we show how to augment a succinct de Bruijn graph 
representation by Bowe et al. (Proc. WABI, 2012) 
to support new operations that let us change order on the fly, effectively
representing all de Bruijn graphs up to some maximum order $K$ in
a single data structure. 
Our experiments show our variable-order de Bruijn graph only modestly increases
space usage, construction time, and navigation time compared to a single order graph.
